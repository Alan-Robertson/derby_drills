\subsection*{Cutting In}
\label{drill:line_to_line_jamming:bean_dip}

This drill requires a blocker

Starting from lanes 1.5 or 3.5 accellerate towards the opposite line.
If starting from stationary, the initial push may involve a toe stop, the rest of the movement must be on skates.

The blocker should accellerate towards the outside line, ready to engage with their shoulders, or backwards.
The jammer should attempt to turn upwards along the line and dip underneath the blocker and clear them completely.
The jammer may turn their shoulders slightly to present their back as an illegal target zone during this movement.

If the jammer feels that they cannot manage that they should attempt to juke back past the blocker's committed hit towards the center of the track.
The jammer should avoid back blocks from turning into the blocker, or low blocks from engaging too low.   
Blockers should avoid performing a high block or a back block on the jammer.
Dipping slightly early to give warning of the intended directory



%\begin{figure}[h]
%\begin{center}
%\includegraphics[]{christmas_tree/chr_1}
%\caption{Path of the carve.\label{fig:chr_1}}
%\end{center}
%
%\end{figure}


{\it Progression:} Incorporate an approach with a juke - the bean dip is the alternative for running the opposite line.   
