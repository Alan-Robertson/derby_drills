\subsection*{Juke Approach}
\label{drill:line_to_line_jamming:juke_approach}

Starting from lanes 1.5 or 3.5 accellerate towards the opposite line.
If starting from stationary, the initial push may involve a toe stop, the rest of the movement must be on skates.
If starting with a roll then aim to start in a fowards direction and carve suddenly towards the opposite line.

When approaching the wall, throw in a juke to move the wall away from the intended path.
Then either cut back towards the closer line, or continue towards the opposite line before carving in. 
This path may be seen in figure \ref{fig:chr_1}.

The drill may be performed with cones or chairs representing blockers. 

\begin{figure}[h]
\begin{center}
\includegraphics[]{christmas_tree/chr_2}
\caption{Path of both the carve and the possible juke.\label{fig:chr_1}}
\end{center}

\end{figure}


{\it Progression:} Include a pair of blockers who will track the incoming jammer. The blockers will attempt to remove the jammer from the track, while the jammer should aim to break through the seam, or end engaged pushing back towards the center.  

The jammer should seek to juke the blockers with the initial movement, or failing that should seek to challenge the blockers sufficiently in the rush to the line to pass them there.

For further progressions the jammer may incorporate additional jukes on the approach. 
