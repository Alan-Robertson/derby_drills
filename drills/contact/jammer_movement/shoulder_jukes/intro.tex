\section*{Shoulder Jukes}
\label{sec:jammer_movement/shoulder_jukes}

This section details juking skills involving the upper body.
These skills require the dissassociation of the skater's hip and leg movements from their shoulders for success.
This independent movement can be developed using footwork drills such as \Cref{sec:sticky/reverse_carves}, \Cref{sec:footwork/russian_circles}, \Cref{sec:footwork/finnish_drill}.  


These jukes have two main ideas - the first is to divide the blockers attention between the facing of the jammer's hips, and the facing of the jammer's shoulders. 
Jammers should practice exiting from each of these movements in both directions - if the hips or shoulder indicates their true trajectory then the juke will be neutralised.  
Secondly, by adopting a torsioned position the jammer should be able to snap either their upper half or lower half of their body to their true trajectory.
: 
