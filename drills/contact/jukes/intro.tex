\chapter{Jukes}

This chapter will cover a collection of juking drills, primarily for jammers.


The core philosophy of the juke is that the jammer should maintain at least two options for movement at all points in time.  
Ideally the blockers should not be able to determine what the jammer's next action will be, and are forced to pre-emptively react.    
If the jammer has maintained their options then the blocker's pre-emptive reaction will leave them out of position, and hopefully overcommitted.  


As a result, each juke drill in this chapter will practice at least two movements, with the idea that the jammer should be prepared to transpose between either based on how the blocker's react.

Jukes will then be broken up into a `main line', which typically will be the telegraphed motion, and then a `secondary line', representing the transposition.    
