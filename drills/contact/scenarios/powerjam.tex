\section{Power Jam Scenarios}
\label{sec:scenarios/power_jam}

Starting with one jammer in the box. 

There are three variants of this drill depending on how long the jammer is in the box for.

\begin{enumerate}
\item The jammer is in for a short period of time (typically less than 10 seconds).  
\item The jammer is sitting, but only just (10-20 seconds) 
\item The jammer is in the box for close to, or the full 30 seconds. 
\end{enumerate}



In the first case the goal for the `power-ful' blockers is to perform at most one quick offensive move, then reset to intercept the returning jammer. 
The defending blockers will try to prevent that first move before settling into regular derby.   


In the second case the goal is to get the jammer out, and reset, with a clean pass the power-ful jammer can hope to be re-entering the engagement zone as the power-less jammer is returning from the box.
The defending blockers may wish to     



In the last case the blockers have free reign for typically at least the initial pass and one scoring pass.
They should consider how best to set their jammer up for points on their first scoring pass, and reset when required to catch the returning jammer.       


Defensive blockers 

{\it Progressions}
Alternatively the time that the jammer is in the box may be randomised, the jammer may wish to attempt to communicate their time remaining to their own team to help with defensive options. 
