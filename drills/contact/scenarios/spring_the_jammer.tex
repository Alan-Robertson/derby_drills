\section{Spring the Jammer}
\label{drill:scenarios/spring_the_jammer}

This drill trains the offence to defence switch and back again.
It runs on the assumption that one team's jammer has just gained lead, and the other team needs to perform a fast offence before setting up to receive the opposing jammer. 


Requires two packs, one in black one in white.

\begin{figure}
\centering
\includegraphics[]{img/spring_the_jammer/track}
\caption{Spring the Jammer drill setup. \label{drill:scenarios/spring_the_jammer}}
\end{figure}

{\bf Drill:}
One jammer starts engaged on a back wall on the jammer line, the other starts unengaged just past the pivot line. 
This setup can be seen in \Cref{drill:scenarios/spring_the_jammer}. 

The drill begins with the front jammer gaining lead.
The back jammer's team should seek to perform a large offensive manoeuvre to `spring' their jammer. 


The drill ends when the lead jammer calls it off. The lead jammer's goal is to maximise their own points, while ensuring the other jammer scores no points.  



{\bf Concepts:}

Once the back team's jammer has gained lead, those blockers should be aware that offence is imminent.  
One option for this team is to hit the jammer out, and use the offence to stretch pack backwards along the track, elongating the run-back.



If the jammer is driving hard, rather than allowing the offence to break apart the wall, the wall might consider using the momentum from the drive to overshoot the offence while still holding as a three wall.    


{\bf Progressions:}

Move the front jammer back behind a two wall of opposing skaters. Lead is called when either jammer escapes. 
The setup for this drill can be seen in \Cref{drill:scenarios/spring_the_jammer_progression}. 

\begin{figure}
\centering
\includegraphics[]{img/spring_the_jammer/progression}
\caption{Spring the Jammer drill setup with a front two-wall. \label{drill:scenarios/spring_the_jammer_progression}}
\end{figure}


