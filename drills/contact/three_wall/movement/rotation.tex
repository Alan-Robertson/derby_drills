\subsection*{Drill: Three Wall Rotation} 
\label{drill:three_wall/rotation}

A drill for developing three wall rotation movements.
Start with three blockers. The wall should seek to move up the track at a slight angle, while rotating. 

In each case the brace should become the new butt on the line, while the previous butt that was holding the line should become the new brace.
The blocker closest to the centre of the track will be the pivot about which the wall rotates, and will remain in their relative position.

This can be seen in figure \ref{fig:three_wall_rotation}.

\begin{figure}[h]
\begin{subfigure}{0.49\linewidth}
\centering
\includegraphics[]{img/three_wall_rotation/first_move}
\caption{Inside to Outside}
\end{subfigure}
\begin{subfigure}{0.49\linewidth}
\centering
\includegraphics[]{img/three_wall_rotation/second_move}
\caption{Outside to Inside}
\end{subfigure}
\caption{Three Wall Rotation Movements\label{fig:three_wall_rotation}}
\end{figure}

{\bf Common Pitfalls:}
The blocker who swings out along the line will invariably spend part of that movement in an anti-derby direction.  
This leads to the risk of a direction penalty being issued to that blocker.
