\section{Jammer Ping Pong}
\label{sec:three_wall/ping_pong}

This drill requires at least three blockers, and at least three jammers.

The focus of the drill is the wall knowing when to hold, control and release jammers, and when to reset and change focus. 
Jammers may simply jam.


Jammers begin with two jammers behind the jammer line, and one jammer behind the pivot line.
At the start of the drill one jammer from the jammer line is released.

The blockers should attempt to block the jammer as per usual derby rules.

When the jammer reaches the pivot line, that jammer stops, and the jammer at the pivot line is released. 
The next jammer is then blocked in anti-derby direction back to the jammer line.
This is then repeated with the jammers cycling. 
The drill may be run from anywhere from 2-5 minutes depending on the stamina of the blockers. 

If the jammers are having too much success, then additional blockers may be added.


{\color{red} TODO: Figure}

\subsection*{Variant: No Hitting Out}  
\label{sec:three_wall/ping_pong/no_hitting_out}

This variant focuses on controlling juke-y jammers who will gain an advantage if they're hit out instead of controlled on track.  

In this variant if the jammer is hit out then they are done and the next jammer is released immediately.
Blockers should focus on holding the jammer without knocking them out. 


\subsection*{Variant: They Don't Stop Coming} 
\label{sec:three_wall/ping_pong/dont_stop}

This variant focuses on sucking back and intercepting an incoming jammer.
Jammers all start behind the jammer line, and are released from the jammer line. 

Jammers are released when the previous jammer has escaped.

Blockers may extend past the pivot line as if the last opposing blocker on track is at the pivot line. Blockers should not break pack. 

The wall should reform and re-catch the next jammer who will likely have a run-up.  
