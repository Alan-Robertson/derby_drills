\section{Drop Step}
\label{sec:crossovers/drop_step}
A related movement to the backwards crossover. 

This motion generates considerable acceleration without the use of toe stops.
The variant presented in \Cref{fig:drop_step} shows the generation of lateral motion, while later jammer juking drills (\Cref{sec:juke/drop_step_spin}) will use a whole spin to translate some of that acceleration to forwards momentum.    
The backwards momentum version of this is to simply do a backwards crossover.


\begin{figure}
\begin{subfigure}{0.3\linewidth}
\includegraphics[scale=0.25]{img/drop_step/drop_step_1}
\caption{Feet start parallel}
\end{subfigure}
\begin{subfigure}{0.3\linewidth}
\includegraphics[scale=0.25]{img/drop_step/drop_step_2}
\caption{One foot begins by being pulled backwards and squeezed, generating force in the direction of that foot.}
\end{subfigure}
\begin{subfigure}{0.3\linewidth}
\includegraphics[scale=0.25]{img/drop_step/drop_step_3}
\caption{The squeeze continues as the other foot is similarly pulled backwards, this one should only be driven enough to brace the motion of the other foot.}     
\end{subfigure}
\begin{subfigure}{0.3\linewidth}
\includegraphics[scale=0.25]{img/drop_step/drop_step_4}
\caption{The first foot is lifted.}
\end{subfigure}
\begin{subfigure}{0.3\linewidth}
\includegraphics[scale=0.25]{img/drop_step/drop_step_5}
\caption{The remaining foot continues the motion as the skater exits the manoeuvre backwards.} 
\end{subfigure}
\caption{The "pull under" manoeuvre. This particular variant leaves the skater moving laterally relative to their starting position.  \label{fig:drop_step}}
\end{figure}
