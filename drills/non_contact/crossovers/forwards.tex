\section{Forwards Crossovers}
\label{sec:crossovers/forwards}


\subsection*{Developmental Drill: Punt}
This drill involves breaking the motions required for a crossover down into individual leg movements.
While this can be performed for any form of crossover, we will be focusing on the forwards derby-direction version in this description. 


The first of these is a `punt'. The skater should begin with their outside leg as far forwards and crossed in front of them as they feel comfortable. Their inside leg should be bent and `locked' in for this drill. 

The motion should be for a long, low stroke with the skate, pulling from left to right to the outside of the track, lifting at the end of the movement and placing the skate in the same relative starting position. 
Force is generated from the sideways motion of the skate.

Do two to three laps with this motion. 
Skaters may feel that the long hold with the passive leg may be tiring.


\subsection*{Developmental Drill: Pull Under}
This drill involves breaking the motions required for a crossover down into individual leg movements.
While this can be performed for any form of crossover, we will be focusing on the forwards derby-direction version in this description. 

The second motion is the pull under.
The skater should place their inside leg as far to their left as they feel comfortable.
The skater's outside leg should be bent and `locked' for the duration of this dirll.
The motion should be for a long, low stroke pulling under and behind the skater's outside leg.   

Force is generated from the sideways motion of the skate.

Do two to three laps with this motion. 
Skaters may feel that the long hold with the passive leg may be tiring.


\subsection*{Developmental Drill: Lean-To}
The previous developmental drills focused on the motion of the skate in contact with the ground. This drill looks at the flicking motion.     

The flick is technically optional - most of the power is generated from contact with the ground - however it is stylistically correct.  

The skater should lean their left shoulder against a wall, and bend and lock their right leg. 
Next the skater should pull their left leg under as if they were in motion.     

The focus should be on the final flick of the motion, not the generation of force.   
As the skate reaches the end of its motion, lift first the inside two wheels, then drag and flick the outside two wheels.     


\subsection*{Crossovers for Speed}
\label{drill:crossovers/forwards/speed}
This drill is focused on using crossovers for maintaining speed around the track.  

The power driving the crossover comes from the lateral motion of the skate along the floor. 
In this drill the goal is to maintain contact with the floor for the longest period of time possible.
This involves getting and staying low to provide the maximum reach for the legs.  

\subsection*{Crossovers for Acceleration}
\label{drill:crossovers/forwards/accel}
Unlike the speed drill, this involves fast short crossovers as a method of gaining speed quickly.    
Here the focus is on using the crossover motion to kick the ground and quickly flick off it rather than stroking through.  
This is best performed from an initially stopped or slowly rolling position before settling into speed based crossovers for the lap.



\subsection*{Alternating Crossovers}
\label{drill:crossovers/forwards/alternating}
A drill for using crossovers for agility. 

Start on the outside line facing in derby direction.  
\begin{itemize}
\item Perform an accelerative crossover towards the inside  
\item Before reaching the inside line, perform an accelerative crossover towards the outside. 
\item Repeat
\end{itemize}
