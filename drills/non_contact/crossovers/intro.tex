\chapter{Crossovers}
\label{ch:crossovers}

This chapter deals with crossovers.

Crossovers are the most stamina-efficient of maintaining speed around the track.     
Some styles provide large amounts of accelleration, but are often coupled with toe-stop running for the initia.


As crossovers are a common part of warm ups, we will begin with a few warm up drills:


\subsection*{Warm Up Drill: Hot Laps}
A simple drill for skaters who are already proficient at crossovers.
Perform 3-5 laps of crossovers at a moderate to fast pace.


\subsection*{Warm Up Drill: Pyramids}
This is a common cardio and general fitness drill.

Skaters should form pairs.
One member of each pair should perform \{5, 4, 3, 2, 1\} lap of crossovers, while the other performs a non-derby exercise (such as planks, sit ups, push ups, squats etc).    

Once the skating member has finished their laps, the members swap positions.


{\bf Progression:}
Repeat the drill, in mirrored order this time increasing the number of laps.


\subsection*{Warm Up Drill: Balanced Hot Laps}
A more balanced set of cross-overs.

Rolling rest implies that skaters should remain on track and upright, but simply roll.

\begin{itemize}
\item Two minutes of crossovers in derby direction
\item 30 seconds rolling rest in anti-derby direction (still on track)
\item Two minutes of crossovers in anti-derby direction
\item 30 seconds of rolling rest in derby direction.
\item Two minutes of backwards crossovers in derby direction. 
\item 30 seconds rolling rest in anti-derby direction (still on track)
\item Two minutes of backwards crossovers in anti-derby direction
\end{itemize}


\subsection*{Warm Up Drill: Peeling Off} 

Skaters should line up from slowest to fastest.

Begin with the slowest skater performing either one or two laps of crossovers at a comfortable pace (say 60\% of full effort).   
The pack should follow and keep pace with this leading skater.
That skater then peels off and joins the back of the pack. 
The next fastest skater then takes over, at what they would consider to be a comfortable pace.


Skaters that have finished their lap should push themselves to keep up with faster skaters.
If they find that they can no longer keep up, they may go and get a drink.

The drill continues until each skater has led a lap.

{\bf Progressions:} The drill may also be run with backwards crossovers. 
