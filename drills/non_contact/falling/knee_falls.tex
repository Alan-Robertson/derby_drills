\section{Knee Falls}
\label{sec:falls/knee_falls}

A useful skill for falling.

The goal of this skill is to take a knee to regain control. 

Rather than falling directly down, the primary goal should be to burn off speed at an oblique angle. 
The shallower the angle the lessen the impact. 


If the skater reaches a stop then the next question should be how to resume an upright position. 
For stability it may be useful to plant a rear toe-stop.
If they have sufficient thigh strength they should be able to stand up from this position. 
If that is not the case then it is imperative that skaters push up by first placing their hands on their thigh, and not on the track.  
This minimises the risk that another skater runs over their fingers.



\subsection*{Knee Taps}
A controlled knee fall.
Skaters should lower themselves while skating and tap a knee pad to the floor regain control, then rise again to a skating position.  



\subsection*{Sequential Knee Taps}
Technically these are lunges on skates.

Rather than regaining position skaters should instead bring their other knee forward and perform a sequential knee tap.   
This can be repeated around the track.
