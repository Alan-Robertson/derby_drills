% TODO
\section{Stop to Slides}
\label{sec:footwork/stop_to_slide}

This section contains footwork drills relating to moving between stops and toe-stops.
This forms the bais of a range of dynamic jammer movements which are useful in constructing jukes.


\subsection*{Drill: Powerslide to Toe Stops} 
\label{drill:toe_stops/slides/powerslide}
% https://www.youtube.com/watch?v=rMbz9rvgfzw

This is a more advanced drill that requires that the skater can already perform either hockey stops or power slides, see \Cref{sec:stopping/hockey} and \Cref{sec:stopping/power_slides} for drills on these techniques. 

Set up either across both sides of the track, or across the middle of the track. 
Skate from one line to the other and perform either a hockey stop, or a power-slide.
Drop from that stop onto toe-stops to reset.

{\color{red} TODO: Figure}

\subsection*{Drill: Powerslide to Juke} 
\label{drill:toe_stops/slides/powerslide_juke}
% https://www.youtube.com/watch?v=rMbz9rvgfzw

This is an extension of the previous drill.
Set up two chairs representing blockers in lanes 1.5 and 3.5. 

The skater should begin behind the chairs and skate towards a line.   
As they reach the line they should perform a powerslide or hockey stop, and then finalise the stop with their toe stops.
This stop should put them right next to the inside or outside edge of the track.

From this position they should then either perform a toe-stop side shuffle (\cref{sec:toe_stops/side_shuffle}, or directly launch from their toe stop into a jump around the chair.  

To complete the motion they should then skate through the middle of the chairs and back around.
This motion should form a figure 8 pattern.

{\color{red} TODO: Figure}
