\subsection*{Finnish Footwork Drill}
\label{drill:footwork/finnish}

This drill was poached from a Finnish ice hockey drill, hence the name. 

This drill may be run between the outside and inside lines of the track, but having some more room for a run-up is preferred (for example by running it across the inside of the track).

The skater should begin on the inside line while facing the outside line.
Throughout the drill their shoulders and torso should remain facing in this direction. 

\begin{itemize}
\item The skater begins by skating towards the outside line at a slightly oblique angle in derby direction. They may mix a crossover into this motion.
\item They should perform a carve, burning off their forward momentum while conserving their sideways momentum. 
\item From here the skater should perform backwards crossovers to propel themselves back towards the inside line.
\item Once at the inside line they should perform a backwards carve, resetting their position. 
\end{itemize}

As the sequence is repeated the skater will naturally move down the track.

The drill should be repeated in both directions.


\includegraphics[scale=0.25]{img/finnish_drill/path}


{\it Commentary}:
The drill has two main goals - keeping the skater's torso moving independently of their legs, and working on the blurred line between a hockey stop and a carve such that the skater can redirect their momentum efficiently.    


{\it Variant}:
The forwards and backwards carves may be reversed - the backwards carves being performed at the front line and the forwards carves at the back line.
