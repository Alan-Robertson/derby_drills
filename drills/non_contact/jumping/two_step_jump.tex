\subsection{Two Step Jump}

This is an extension of the two step turn. 
The basic version of this jump with the spin preferences the outside line, while eliminating the spin allows it to be used effectively on the inside too.   

The static version starts with the skater in a split stance in lane 1 or lane 4.   
The leg closest to the line should be behind.


The stepping movement is a toe-stop step from the back, then on the front leg, into a jump around the line. 
The initial focus should be on spinning to land back on the initial toe stop, plant the second toe stop and stop dead.

This drill should start with no obstacles except maybe a cone on the line, but should then progress toa chair. 
The chair should initially be seat side out, then back side facing the skater, then back side facing the line.  

Initial Progressions are as follows:
\begin{itemize}
\item Moving the chair closer to the line.
\item Gaining Height on the jump
\item Maintaining momentum from the toe stop landing. 
\item Rolling into the motion.
\item Increasing the distance of the jump.
\end{itemize}

Subsequent progressions focus on improving the form of the jump.
Our goal is to eliminate the spin entirely, for this we need to land with our first toe stop half a lane in from the line, then grape-vine with our other leg behind the first.       

The final motion should allow the skater to continue moving in a forward direction with-out losing momentum to the spin.   

The next progressions are to:

\begin{itemize}
\item Add incoming speed
\item Add distance   
\end{itemize}

Another variant is to stop dead on toe stops from the jump, before cutting hard towards the centre of the track.
Emphasis here is on stopping on the landing and getting into a low position. 
