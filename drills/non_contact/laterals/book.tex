\section{Opening the Book Laterals}

Begin on the line with your feet in a `T' position. The capping bar of the T should run parallel to the line, while the other foot should be pointing away from the line.     

Skateboard push using the forwards facing foot, laterally skating on the other foot.   

Place the raised foot in line with and parallel to, but facing in the opposite direction to the planted foot.   





\subsection*{Developmental Drill: Lateral Balancing}
\label{drill:lateral/book/balancing}

Laterals involve three different movements, each of which can be practiced independently. 

The first issue that new skaters may encounter is that while they may be proficient in one footed glides with their legs parallel, they may not have practiced with their legs akimbo.     
As this utilises different sets of muscles this may require some initial practice.


The goal of this drill is low speed control. 
This is markedly harder than skating on one foot at speed, and so will help develop better technique before looking at speed and power.   



Begin by standing and balancing on one foot facing forwards with the foot turned out 90 degrees.
The other leg should be raised but still facing forwards. 
The skater's hips should be turned towards the turned out foot. 
Practice this on both legs.


\subsection*{Developmental Drill: Lateral One Foot Glide}
\label{drill:lateral/book/glide}

Once the skaters are comfortable with lateral balancing position add a small `skateboard push' using the outside edges of the otherwise raised leg.
This push should have minimal force such that the motion can be halted without requiring any large stomps or difficult techniques.    

This motion is similar to side-surfing. {\color{red} TODO: Ref side-surfing}.
Weight should be heavily over the forwards leg. 
The knee of this leg should be bent out over and possibly past the toes of the same leg.
The knee should also be in line with the foot, if it's misaligned then the skater will turn.



{\it Common Pitfall:}
It is important that the source of the force from this push is from behind the planted foot.
If the force comes from too far forwards, or the skateboard pushing foot rolls this will throw off the balance of the push.
Typically this can be seen when the skater performs the push, remains stable for a few seconds while their pushing leg is in motion, then when they attempt to arrest and return the pushing leg the acceleration on that leg throws off their balance.// 


{\it Common Pitfall:}
If the skater is having issue maintaining stance, it is possibly a positioning issue. Ensure that the knee is out over their toes and that it is aligned with their foot rather than turned or offset.   

\subsection*{Developmental Drill: Static Weight Transfer}
\label{drill:lateral/book/weight_transfer_static}

Having grasped the initial push, the next drill focuses on transferring between the two feet.

As before we begin with performing the motion at a standstill. 

\begin{itemize}
\item Starting with feet in the `T' position, we begin by turning the hips as far as possible in the direction of the turned out foot. 
\item Next raise the forward facing foot. 
\end{itemize}
The speed of the next few motions depends on the hip flexibility of the skater. 
With lower flexibility these should be performed in quick succession, conversely skaters that are more adept at side surfing might perform a side surf at this point.    
\begin{itemize}
\item Place the raised foot inline-with, but facing out and away from the planted foot. 
\item Lift the first foot, keeping it perpendicular to the newly planted foot  and flip the hips towards the newly planted foot.
\end{itemize}


\subsection*{Developmental Drill: Rolling Weight Transfer}
\label{drill:lateral/book/weight_transfer_rolling}

