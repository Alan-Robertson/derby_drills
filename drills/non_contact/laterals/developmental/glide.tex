\subsection*{Developmental Drill: Push and Glide}
\label{drill:laterals/developmental/push_and_glide}

Once the skaters are comfortable with lateral balancing position add a small `skateboard push' using the outside edges of the otherwise raised leg.
This push should have minimal force such that the motion can be halted without requiring any large stomps or difficult techniques.    

Weight should be heavily over the forwards leg. 
The knee of this leg should be bent out over and possibly past the toes of the same leg.
The knee should also be in line with the foot, if it's misaligned then the skater will turn.

The skater's torso should be facing perpendicular to their direction of travel.


{\it Common Pitfall:}
It is important that the source of the force from this push is from behind the planted foot.
If the force comes from too far forwards, or the skateboard pushing foot rolls this will throw off the balance of the push.
Typically this can be seen when the skater performs the push, remains stable for a few seconds while their pushing leg is in motion, then when they attempt to arrest and return the pushing leg the acceleration on that leg throws off their balance. 


{\it Common Pitfall:}
If the skater is having issue maintaining stance, it is possibly a positioning issue. Ensure that the knee is out over their toes and that it is aligned with their foot rather than turned or offset. 

