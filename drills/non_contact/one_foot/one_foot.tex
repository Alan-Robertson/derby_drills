\section{Developing One Footed Skating}
\label{sec:one_foot/development}


This section discusses a number of prepartory drills to develop one footed skating. 
For introductory one foot skating it is enough to lock the thigh and glide forwards.
More complex weight distributions and edge work are a progression from this skill.     


\subsection*{Developmental Drill: Balancing}
The first and simplest predicate for one footed skating is standing on one foot. 

This may be practiced off skates for skaters who have not yet developed their proprioception.    

The skate should be planted and pressure applied evenly to maintain position.   
The skater should not be rolling either forwards or backwards.

\subsection*{Developmental Drill: Lifting}
The next issue that skaters often encounter is that when lifting their leg while skating that this action throws off their balance.
They will manage to keep their balance while their leg continues its initial swing, then arresting that motion will throw off their balance, ending their one footed sojourn.  


Starting stationary, lean out over one hip until one foot naturally leaves the ground.
This circumvents the issues with raising the foot with too much force and places the skater's weight in the correct position for one footed skating.    



Another approach to this is to emphasise lifting from the `hip' rather than from the thigh.
This may make the initial balance distribution slightly harder, but will also help with developing one foot form.


\subsection*{Developmental Drill: Lifting and Rolling}
 

Assume the same position as the lifting drill, this time while rolling with a small amount of momentum. 
While rolling, gently lift the extended leg, allowing the extention to `hitch' the hip.
Keep the grounded leg bent and ready to fall forwards to take that knee if difficulties arise.  


\subsection*{Developmental Drill: Rocking}

Having established that the skater is able to lift into a one footed stance, and that they are able to balance while stationary, the next goal is to get a small amount of dynamic motion. 

From a starting position of balancing on one foot, rock backwards and forwards on that foot. 
The skater will need to adjust their ankle to compensate for the movement.


\subsection*{Warm Up Drill: One Foot Glides} 

Gain some momentum and glide on one foot around the track. 

If this is part of a developmental drill start at the jammer line and begin with a glide to the pivot line. 
You should aim to extend this through to the turn. 
At this stage some basic one footed edge work and leaning will be required to make it through the turn.   
Some reasonable benchmarks are: From a stationary start the skater should make it to either one quarter or halfway around the track. 

From a rolling start the skater should aim to make it around the track.
Eventually the goal should be to extend this to a glide around the track from a stationary start. 

