\section{Developing One Footed Skating}
\label{sec:one_foot/development}


This section discusses a number of prepartory drills to  


\subsection*{Balancing}
The first and simplest predicate for one footed skating is standing on one foot. 

This may be practiced off skates for skaters who have not yet developed their proprioception.    

The skate should be planted and pressure applied evenly to maintain position.   
The skater should not be rolling either forwards or backwards.

\subsection*{Lifting}
The next issue that skaters often encounter is that when lifting their leg while skating that this action throws off their balance.
They will manage to keep their balance while their leg continues its initial swing, then arresting that motion will throw off their balance, ending their one footed sojourn.  


Starting stationary, lean out over one hip until one foot naturally leaves the ground.
This circumvents the issues with raising the foot with too much force and places the skater's weight in the correct position for one footed skating.    


\subsection*{Rocking}

Having established that the skater is able to lift into a one footed stance, and that they are able to balance while stationary, the next goal is to get a small amount of dynamic motion. 

From a starting position of balancing on one foot, rock backwards and forwards on that foot. 
The skater will need to adjust their ankle to compensate for the movement.


\subsection*{Lifting and Rolling}


