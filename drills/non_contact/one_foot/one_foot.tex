\section{Developing One Footed Skating}
\label{sec:one_foot/development}


This section discusses a number of prepartory drills to  


\subsection*{Developmental Drill: Balancing}
The first and simplest predicate for one footed skating is standing on one foot. 

This may be practiced off skates for skaters who have not yet developed their proprioception.    

The skate should be planted and pressure applied evenly to maintain position.   
The skater should not be rolling either forwards or backwards.

\subsection*{Developmental Drill: Lifting}
The next issue that skaters often encounter is that when lifting their leg while skating that this action throws off their balance.
They will manage to keep their balance while their leg continues its initial swing, then arresting that motion will throw off their balance, ending their one footed sojourn.  


Starting stationary, lean out over one hip until one foot naturally leaves the ground.
This circumvents the issues with raising the foot with too much force and places the skater's weight in the correct position for one footed skating.    


\subsection*{Developmental Drill: Lifting and Rolling}
 

Assume the same position as the lifting drill, this time while rolling with a small amount of momentum. 
While rolling, gently lift the extended leg, allowing the extention to `hitch' the hip.
Keep the grounded leg bent and ready to fall forwards to take that knee if difficulties arise.  


\subsection*{Developmental Drill: Rocking}

Having established that the skater is able to lift into a one footed stance, and that they are able to balance while stationary, the next goal is to get a small amount of dynamic motion. 

From a starting position of balancing on one foot, rock backwards and forwards on that foot. 
The skater will need to adjust their ankle to compensate for the movement.


\subsection*{Warm Up Drill: One Foot Glides} 

Gain some momentum and glide on one foot around the track. 
Some reasonable benchmarks are: From a stationary start the skater should make it to either one quarter or halfway around the track. 

From a rolling start the skater should make it around the track.

\subsection*{Warm Up Drill: One Foot Weaves} 

A warm up drill, but also a developmental drill for one footed self-propelled.

Take two lanes, moving at roughly 45$\deg$ on one foot turn to weaves within those lanes.    

\subsection*{Warm Up Drill: One Foot Self Propelled} 

As with one foot weaving, but now additional motion should be generated on the turns to maintain momentum.  

Aim to complete one lap of the track on each foot.

Further development is to not use the extra leg as ballast to help generate momentum, but to only use the leg that is in contact with the ground. 

\subsection*{Warm Up Drill: Serpentine} 

As different extension to one footed weaves.

Rather than two lanes of the track at a sharper angle than 45 degrees glide from line to line on one foot. At the line, or as close to it as the skater feels comfortable use the edges of the skate to turn, as in a weave or a carve to face the other line and continue the motion.   

As we're not currently generating momentum from this motion after two or three turns change feet. 
The focus is on the sharpness of the turn, rather than the speed. 
Speed may be added after technique has been polished.

