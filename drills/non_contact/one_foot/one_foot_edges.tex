\section*{One Foot Edge-Work}

In the previous section the drills were focused on gliding on one foot in a straight line.  
Here we will discuss the development and drills focused on engaging edges on one foot. 

An important part of one foot edge work is the appropriate engagement of hip muscles.
While simple glides can be performed by locking the thigh in place, more controlled motion of the edges of the skate requires engaging either the outside hip muscles, or the inside thigh to core muscles.

The outside hip muscle engagement is similar to `hitching' a hip.


\subsection*{One Foot Cuts}

In small movements on the spot, attempt to perform a `C' cut with one foot, then change feet. 



\subsection*{Warm Up Drill: One Foot Weaves} 

A warm up drill, but also a developmental drill for one footed self-propelled.

Take two lanes, moving at roughly 45$\deg$ on one foot turn to weaves within those lanes.    

\subsection*{Warm Up Drill: One Foot Self Propelled} 

As with one foot weaving, but now additional motion should be generated on the turns to maintain momentum.  

Aim to complete one lap of the track on each foot.

Further development is to not use the extra leg as ballast to help generate momentum, but to only use the leg that is in contact with the ground. 

\subsection*{Warm Up Drill: Serpentine} 

As different extension to one footed weaves.

Rather than two lanes of the track at a sharper angle than 45 degrees glide from line to line on one foot. At the line, or as close to it as the skater feels comfortable use the edges of the skate to turn, as in a weave or a carve to face the other line and continue the motion.   

As we're not currently generating momentum from this motion after two or three turns change feet. 
The focus is on the sharpness of the turn, rather than the speed. 
Speed may be added after technique has been polished.
