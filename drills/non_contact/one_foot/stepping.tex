\section{Stepping}
While not technically skating on one foot, this is a developmental skill for weight transfer that help with one footed skating and techniques that depend on it. For example laterals, transitions etc.   


\subsection*{Technique}

Stepping technique varies between the direction of motion. 

The first concern is to have sufficient static balance when standing (i.e. not rolling) and lifting and placing skates.  
Preventing rolling is a matter of weight distribution.
The rhythm to achieve is to weight up one skate, lift the other, place it, transfer weight to the other skate, lift the first skate, place it and repeat. 


Stepping side to side is easier, as the motion of the movement is orthogonal to either direction of rolling. 

Stepping forwards and backwards may be improved by turning the skates slightly inwards or outwards (different skaters have reported opposite methods as being useful), to get a slight edge to bind against.   
Skaters may also find that weighting up the inside or outside edges improves stability.



\subsection*{Developmental Drill: Stepping in Place} 
\label{drill:one_foot/stepping/in_place}
Without moving, this drill is simply about raising and placing skates back on the ground without rolling.


{\it Common Pitfalls}
Skaters may raise their hips when lifting their leg, unweighting their loaded leg and rolling. This is particularly prevalent on the forwards and backwards steps. 
One approach to ameliorating this is encouraging them to swing their leg from their hip rather than lifting it when placing their foot either forwards or backwards. 


\subsection*{Warm Up Drill: Stepping} 
\label{drill:one_foot/stepping/warm_up}

This is perhaps best concieved as an 80s aerobics class.

One skater leads the drill, the other skaters mirror the `instructor'.
The instructor indicates a direction of stepping, either forwards, backwards, left or right for the skaters to follow.


With more developed groups of skaters synchronisation of steps may be added to the drill. This is not reccomended for developing skaters who should be free to set their own pace and level of comfort.

