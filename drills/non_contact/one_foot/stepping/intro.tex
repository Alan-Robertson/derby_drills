\section*{Stepping}
\label{sec:one_foot/stepping}

While not technically skating on one foot, this is a developmental skill for weight transfer that help with one footed skating and techniques that depend on it. For example laterals, transitions etc.   

\subsection*{Technique}

Stepping technique varies between the direction of motion. 

The first concern is to have sufficient static balance when standing (i.e. not rolling) and lifting and placing skates.  
Preventing rolling is a matter of weight distribution.
The rhythm to achieve is to weight up one skate, lift the other, place it, transfer weight to the other skate, lift the first skate, place it and repeat. 


Stepping side to side is easier, as the motion of the movement is orthogonal to either direction of rolling. 

Stepping forwards and backwards may be improved by turning the skates slightly inwards or outwards (different skaters have reported opposite methods as being useful), to get a slight edge to bind against.   
Skaters may also find that weighting up the inside or outside edges improves stability.


