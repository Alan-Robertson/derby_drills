
\section{Backwards Carving}

\label{sec:sticky/backwards_carves}

The backwards version of a carve.

As with the forwards version of the carve, weight may be distributed over either the forwards or backwards leg. With that said, typically the backwards leg is easier.  

Starting by skating backwards in a split stance the skater draws their inside leg (relative to the track) through the turn first, following with their outside leg.   
The legs should leave the turn co-linearly. 


{\it: Common Pitfalls}
Skaters that are weighting up their back leg may neglect their front edges. If this is happening then their front foot will fail to follow their back, leaving their back leg over-extended behind them.   


Skaters that are not drawing their back leg behind their front leg may find themselves falling over.  


\subsection*{Developmental Drill: Backwards Circles}

This drill helps develop the stance needed for a backwards carve, without worrying about changing direction or other aspects of movement.  


\subsection*{Warm Up Drill:}
A common warm-up drill is to backwards carve within lanes or from line to line in derby direction, alternating leading leg with each turn.    

Emphasis should be on the maintaining form and the sharpness of the turn. For the purposes of technique sharper angled carves should be focused on over forwards speed. 


{\bf Progressions:}

\begin{itemize}
    \item Progressing with carving is a matter of increasing lateral speed and taking sharper angles without hockey stopping. 
    \item  Another progression is to move enough weight to the edges that only two wheels on each skate make contact with the ground.
    \item Foregoing some of the purpose of the drill, with enough speed the carves can be replaced with partial hockey stops. Rather than coming to a complete stop the goal should be to burn off speed until the stop can be skated out of in a carve.    
\end{itemize}


\subsection*{Cross-over Carves}

A variant on the previous drill. 
While carving line to line skaters should perform a backwards crossover in between each carve in the direction of motion.

\begin{itemize}
    \item Carve on the outside line
    \item Backwards Crossover towards the inside line 
    \item Carve on the inside line
    \item Backwards Crossover towards the outside line
    \item Repeat 
\end{itemize}

This will increase the lateral speed with which the skater enters the carve, and by extension puts more pressure on the edge-work required.      

\subsection*{Turn-About Carves}

A variant on the warm up drill.

Rather than carving inwards, instead take the larger outwards carve, looping back in to the center of the track. 

The path taken along the track can be seen in \Cref{fig:sticky/carves/turn_about}, in this instance the skater will be moving along this path backwards. 
