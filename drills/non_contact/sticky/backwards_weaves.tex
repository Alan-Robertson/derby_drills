\section{Backwards Weaves}
\label{sec:sticky/backwards_carves}

The backwards version of a weave. 
The usual concepts of backwards skating apply here; weight in the toes and steering from the heels. 
The skater should lean slightly forwards to abrogate the chance of falling backwards.

As with weaves the momentum comes from the hips, and the goal is to warm up the skater's edges.   
Motion should not be forced, but rather roll through the edge.
Skates, knees and hips should be canti-levered.


{\it Common Pitfall:}
With the weight shifted to the toes and a sharp enough edge being used, there's the tendancy for one skate to fall forwards onto just the front truck - almost like a manual. 
This indicates that the skater is not distributing their weight through the edges on both skates and is just loading one at a time.  
The skater should attempt to keep all wheels in contact with the ground, and use edges on both skates.


\subsection*{Developmental Drill: One Leg at a Time}
Lock in one leg, backwards eggshell with the other leg and then alternate.

From this initial faux weave the goal is to then move first into engaging the other leg in the turn, then discard the drag of the eggshell for alternating edge pressure. 
