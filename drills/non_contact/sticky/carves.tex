\section{Carving}
\label{sec:sticky/carves}

\subsection*{Technique}

Carving is where the skater's skates are nearly co-linear with the weight distributed either over the front leg, or back leg.
Working on carves is a good idea to improve edge-work in skating.  


When turning on the side of the front leg the skater should lean into the turn.  
When turning on the side of the read leg the skater should pre-emptively bring the rear leg forward through the turn and switch leading legs as they lean into the turn.  
When turning the skater's weight should be on the edge closest to their body.   


Typically front leg weighted carving is used to challenge another skater for space, and is heavily practiced by blockers.     
Conversely, back leg weighted carving is generally considered to be less committal and lends itself more easily to disengaging and juking.



{\bf Common pitfalls:}

Developing skaters may be concerned about their loss of momentum on the turn. 
Momentum in the new direction is gained by the cutting motions in the turn, and dropping weight into the turn.   
Both of these will develop with time as the skater's sticky skating improves and should not be a cause for concern.


Not pulling the trailing leg through the turn early enough may lead to the skater exhibiting a loss of balance and a highly oblique angle.   


\subsection*{Warm Up Drill}

A common warm-up drill is to carve within lanes or from line to line in derby direction, alternating leading leg with each turn.    

Emphasis should be on the maintaining form and the sharpness of the turn. For the purposes of technique sharper angled carves should be focused on over forwards speed. 

\begin{itemize}
\item $n$ laps carving, 1 lane 
\item $n$ laps carving, 2 lanes
\item $n$ laps carving, line to line (4 lanes)
\end{itemize}


{\bf Progressions:}

\begin{itemize}
    \item Progressing with carving is a matter of increasing lateral speed and taking sharper angles without hockey stopping. 
    \item  Another progression is to move enough weight to the edges that only two wheels on each skate make contact with the ground.
    \item Foregoing some of the purpose of the drill, with enough speed the carves can be replaced with partial hockey stops. Rather than coming to a complete stop the goal should be to burn off speed until the stop can be skated out of in a carve.    
\end{itemize}


\subsection*{Cross-over Carves}

A variant on the previous drill. 
While carving line to line skaters should perform a crossover in between each carve in the direction of motion.

\begin{itemize}
    \item Carve on the outside line
    \item Crossover towards the inside line 
    \item Carve on the inside line
    \item Crossover towards the outside line
    \item Repeat 
\end{itemize}

This will increase the lateral speed with which the skater enters the carve, and by extension puts more pressure on the edge-work required.      
