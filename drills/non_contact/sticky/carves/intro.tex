\section{Carving}
\label{sec:sticky/carves}


\begin{figure}
\centering
\includegraphics[scale=0.25]{img/carves/carves_1}
\caption{Approximate path of the carve. Note that in the turn the back leg is brought through and exits before the front leg. \label{fig:carves}}
\end{figure}

\subsection*{Technique}

Carving is where the skater's skates are nearly co-linear with the weight distributed either over the front leg, or back leg.
A figure of the approximate arc of a carve can be seen in \Cref{fig:carves}.


Working on carves is a good idea to improve edge-work in skating.  


When turning on the side of the front leg the skater should lean into the turn.  
When turning on the side of the read leg the skater should pre-emptively bring the rear leg forward through the turn and switch leading legs as they lean into the turn.  
When turning the skater's weight should be on the edge closest to their body.   


Typically front leg weighted carving is used to challenge another skater for space, and is heavily practiced by blockers.     
Conversely, back leg weighted carving is generally considered to be less committal and lends itself more easily to disengaging and juking.


{\bf Common pitfalls:}
Skaters may attempt to turn via lifting their front wheels and correcting their position.
This comes from attempting to turn via pulling the foot rather than putting the weight through their edges. \\ 



Developing skaters may be concerned about their loss of momentum on the turn. 
Momentum in the new direction is gained by the cutting motions in the turn, and dropping weight into the turn.   
Both of these will develop with time as the skater's sticky skating improves and should not be a cause for concern. \\ 

Skaters may focus on their front leg, leaving their trailing leg at a perpendicular angle to their direction of motion.   
Remind them to bring their trailing leg through the turn and that edges on both skates need to be engaged .
