\section{Eggshells}
\label{sec:sticky/eggshells}

\subsection*{Technique}

Eggshells technique is focused around stance and weight distribution.
To move in a forwards direction weight should be centered two-thirds back in the skater's foot, typically towards the front of the heel. 




\subsection*{Development Drill}

\subsection*{Warm Up Drill}

\subsection*{One Edge at a Time}
\label{drill:sticky/eggshells/one_edge}

This is a variant drill for regular sticky skating. 
It's targeted skaters who are comfortable moving around on skates, but may be compensating for not fully developed edge-work in sticky skating.
Breaking down the motions and forcing each to pull their own weight should show this up quite quickly.

The core idea of the drill is to only use a single movement for propulsion.  

One leg should be stationary, the other should begin on the in, push out as you would when sticky skating then lift the leg at the end of the motion.  
Place the leg back on the inside and repeat. 

For the alternative motion the leg should start on the out and pull in.   

Avoid large flicks and other crossover-like motions.    

Repeat the following skating forwards and backwards. 
\begin{itemize}
    \item 1$\times$ lap right leg pushing out, left leg stationary  
    \item 1$\times$ lap left leg pushing out, right leg stationary 
    \item 1$\times$ lap right leg pulling in, left leg stationary 
    \item 1$\times$ lap right leg pulling in, right leg stationary 
\end{itemize}

To put the motions back together:
\begin{itemize}
    \item 2$\times$ laps regular sticky skating. 
\end{itemize}
