\section{Eggshells}
\label{sec:sticky/eggshells}

Also called `melons', `bubbles', `lemons' and likely a small plethora of other names. 


\subsection*{Technique}

Eggshell technique is focused around stance and weight distribution.
To move in a forwards direction weight should be centered two-thirds back in the skater's foot, typically towards the front of the heel, or right in front of the back axle of the skate.  
This will vary slightly from skater to skater, and will be adjusted depending on the position of their upper body. 

The skater's skates should move outwards from their body via application of weight to their outside edges, then move back inwards via application of weight to their inside edges.
The movement can be seen in \Cref{fig:sticky/eggshells}.

\begin{figure}
\begin{center}
\begin{subfigure}{0.19\linewidth}
\includegraphics[scale=0.17]{img/eggshells/eggshell_1}
\end{subfigure}
\begin{subfigure}{0.19\linewidth}
\includegraphics[scale=0.17]{img/eggshells/eggshell_2}
\end{subfigure}
\begin{subfigure}{0.19\linewidth}
\includegraphics[scale=0.17]{img/eggshells/eggshell_3}
\end{subfigure}
\begin{subfigure}{0.19\linewidth}
\includegraphics[scale=0.17]{img/eggshells/eggshell_4}
\end{subfigure}
\begin{subfigure}{0.19\linewidth}
\includegraphics[scale=0.17]{img/eggshells/eggshell_5}
\end{subfigure}
\caption{Eggshell motions. The darker wheels indicate where weight should be distributed between the edges of the skates. 
\label{fig:sticky/eggshells}
}
\end{center}
\end{figure}


The skater should be using just their lower body to perform this motion.
By bending their legs they can offset the extension of their legs with the outwards push and maintain a constant height without ``bobbing" up and down. 

{\bf Common Pitfalls: }
If the skater is bobbing up and down as they perform eggshells then they are not bending their knees enough. 

If the skater is lifting or correcting foot position then they are not applying their weight to the edges of their skates.  


If the skater is juttering backwards and forwards as they attempt to perform the eggshell, then their weight distribution is in the centre of their skate, not shifted back towards the heel.   


\subsection*{Variant: Wide Eggshells}
\label{sec:sticky/eggshells/wide}
As with the regular eggshell, except this time the skater should attempt to take a wider trajectory.    
As this will involve a greater leg extension, and we are still trying to avoid bobbing up and down, this will require a lower initial stance.  


\subsection*{Variant: Figure 8}
\label{sec:sticky/eggshells/figure_8}

The outwards push is identical to a regular eggshell, except this time the inwards pull should be staggered such that one skate ends up in front of the other.  
The skate in front should alternate between eggshells. 


\subsection*{Variant: Fast Eggshells}
\label{sec:sticky/eggshells/fast}

In this we avoid pushing the eggshell fully out or in, instead the skate oscillates on the inside and outside of the knee.  
Weight should be distributed more towards the back truck such that the front is unweighted for faster steering.   


\subsection*{Developmental Drill: One the Spot}
This drill is about getting comfortable enough with the edges of the skates to perform the motions required for an eggshell.   

It also assists in shifting weight from the front to the back of the skate. 

Starting from a stopped position, the skater should attempt to perform one eggshell motion forwards, then the same motion backwards, ending up in their starting position.  


{\it Common Pitfalls:} As this is likely one of the first drills that a new skater will attempt, this is also where you may pick up that a skater's trucks are too tight, and are impeding their ability to turn their skate. 
 
\subsection*{Development Drill: One Edge at a Time}
\label{drill:sticky/eggshells/one_edge}

This is a variant drill for regular sticky skating. 
It's targeted skaters who are comfortable moving around on skates, but may be compensating for not fully developed edge-work in sticky skating.
Breaking down the motions and forcing each to pull their own weight should show this up quite quickly.

The core idea of the drill is to only use a single movement for propulsion.  

One leg should be stationary, the other should begin on the in, push out as you would when sticky skating then lift the leg at the end of the motion.  
Place the leg back on the inside and repeat. 

For the alternative motion the leg should start on the out and pull in.   

Avoid large flicks and other crossover-like motions.    

Repeat the following skating forwards and backwards. 
\begin{itemize}
    \item 1$\times$ lap right leg pushing out, left leg stationary  
    \item 1$\times$ lap left leg pushing out, right leg stationary 
    \item 1$\times$ lap right leg pulling in, left leg stationary 
    \item 1$\times$ lap right leg pulling in, right leg stationary 
\end{itemize}

To put the motions back together:
\begin{itemize}
    \item 2$\times$ laps regular sticky skating. 
\end{itemize}


\subsection*{Warm Up Drill: Eggshells and Variants}

This is a collection of egg-shell drills that can be performed as part of a warm-up. 

\begin{itemize}
    \item Regular eggshells  
    \item Wide eggshells
    \item Eggshells with alternating legs crossing in front and behind (figure 8) 
    \item Short fast eggshells, skates coming into proximity 
    \item Squat position eggshells, skates only sliding    
\end{itemize}
