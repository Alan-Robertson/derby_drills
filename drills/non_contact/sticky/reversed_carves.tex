\section{Reversed Carves}
\label{sec:sticky/reverse_carves}

Typically the stance of a carve leans such that the direction of turning, the skates and the edges are all aligned.    
Here we will generalise these assumptions. 


Using the same split stance as a carve, manipulate your edges to turn in the opposite direction. 


Highly oblique angles are fine for this, unlike a regular carve where the drop and turn affords the skater the ability to gain additional momentum, for this technique the turn comes from edgework, necessarily trading forwards momentum to both burn off the initial lateral momentum, and turning to provide the opposite lateral momentum.    

\subsection*{Serpentine Reversed Carves}
\label{drill:sticky/reverse_carves/serpentine}

Begin skating forwards in a split stance.

Start a carve towards either the out or inside, then without changing stance turn the carve towards the opposite line. 

Regain momentum and repeat.
