\section{Weaving}
\label{sec:sticky/weaves}

\subsection{Technique}

Weaving involves forwards motion on skates while the skates remain approximately parallel.   

A simple version is to syncopate the motions of a basic eggshell.  
One leg is pushing out while the other is dragging in, then the legs swap roles. 

Emphasis should be on the form of the skater's hips, shoulders, ankles and knees. 
In particular the skater's shoulders should maintain an even height, while their hips, knees and ankles should oscillate laterally with hips and ankles aligned and knees out of alignment.    

Speed is generated through the egg-shell-like cuts performed by each leg.   

\subsection{Warm Up Drill}


A common warm-up drill is to weave within lanes or from line to line in derby direction.

Emphasis should be on the maintaining form and the sharpness of the turn. 
For the purposes of technique, sharper angled weaves should be focused on over forwards speed.
Maintaining parallel skates is also important, otherwise weaves blur into carves. 

\begin{itemize}
\item $n$ laps carving, 1 lane 
\item $n$ laps carving, 2 lanes
\item $n$ laps carving, line to line (4 lanes)
\end{itemize}


{\bf Progressions:}
Progressing weaving involves increasing speed while maintaining form.   
Speed is generated by increased force through the edges.  

Another approach is to encourage skaters to weave using only two wheels on each skate. This places greater emphasis on edge-work. 
