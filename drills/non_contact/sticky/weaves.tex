\section{Weaving}
\label{sec:sticky/weaves}

Weaving is a useful skill for developing and warming up to edgework. 

\subsection*{Technique}

Weaving involves forwards motion on skates while the skates remain approximately parallel.   

A simple version is to syncopate the motions of a basic eggshell.  
One leg is pushing out while the other is dragging in, then the legs swap roles. 

Emphasis should be on the form of the skater's hips, shoulders, ankles and knees. 
In particular the skater's shoulders should maintain an even height, while their hips, knees and ankles should oscillate laterally with hips and ankles aligned and knees out of alignment.    
Hips should turn sharply with each weave, and should feel like the skater is `throwing' their hips. 


Speed is generated through the egg-shell-like cuts performed by each leg, and by the skater dropping their weight with each turn.  

\subsection*{Developmental Drill Sequence: Reconstruction}

After eggshells, developing skaters may not yet have sufficient control to be doing different things with each leg.       

This drill sequence tries to break down and re-build weaves as a series of simple motions.   

\subsubsection*{Serial D Cuts}
Begin with locking one leg in the forwards position.
Perform an `eggshell' like maneouvre with the free leg, starting by {\bf pushing out} then pulling in.
Swap which leg is locked, and repeat.


\subsubsection*{Turning}
As with the previous component, but now bring the free foot further through the turn such that the skater turns with each motion rather than simply rolling forwards. This will require bringing the inside leg through the turn slightly.    
Focus on turning the hips to face and pull through the turn.

\subsubsection*{Internal D Cuts}
Begin with locking one leg in the forwards position.
Perform an `eggshell' like maneouvre with the free leg, starting by {\bf pulling in}, then pushing out.
Swap which leg is locked, and repeat.

\subsubsection*{Internal Turning}
As with the previous component, but now bring the free foot further through the turn such that the skater turns with each motion rather than simply rolling forwards. This will require bringing the outside leg through the turn slightly.    
Focus on turning the hips to face and pull through the turn.

\subsubsection*{Composition}
We now combine the motions of each foot, hopefully having keyed this in part to the hip movements.   
The goal here is to get the synchronisation and rhythm down having developed the components. 


\subsection*{Warm Up Drill}
\label{drill:sticky/weaves/warm_up}

A common warm-up drill is to weave within lanes or from line to line in derby direction.

Emphasis should be on the maintaining form and the sharpness of the turn. 
For the purposes of technique, sharper angled weaves should be focused on over forwards speed.
Maintaining parallel skates is also important, otherwise weaves blur into carves. 

\begin{itemize}
\item $n$ laps carving, 1 lane 
\item $n$ laps carving, 2 lanes
\item $n$ laps carving, line to line (4 lanes)
\end{itemize}


{\bf Progressions:}
Progressing weaving involves increasing speed while maintaining form.   
Speed is generated by increased force through the edges.  

Another approach is to encourage skaters to weave using only two wheels on each skate. This places greater emphasis on edge-work. 
