\subsection{One Foot Hockey Stops}

\subsubsection{Developmental: One Foot Circles}
One foot hockey stops are best approached via the `question-mark' method.

Start with single foot circles, working turning `inwards' over the other leg.

Once skaters are comfortable with this movement, their next goal is to burn off all speed during this turn.
This will involve varying the pressure on the edge as they turn, blurring the distinction between their turn and a one-footed plough stop.
Throughout all these progressions it is essential that the skater holds their balance after coming to a stop - toppling or placing their other foot down will teach poor habits and form.  


\subsubsection{Developmental Drill: Carving In}

The next progression is to begin skating forwards on one leg at low speed, then to carve in, coming to a stop. 

From here it's a matter of increasing speed slowly and performing the same manoeuvre.
As speed increases, the rate of the turn will also have to increase to compensate.

At a particular point this will also require that the turn ends with the foot perpendicular to the motion of travel and sliding to burn off speed. 


% TODO  
