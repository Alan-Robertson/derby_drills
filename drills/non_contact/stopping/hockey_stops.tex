\section{Hockey Stops}
\label{sec:stopping/hockey}

There are two rough approaches to hockey stopping: the carve into the stop, and a single sudden sharp turn.   

\subsection*{Developmental Drill: Carving In} 

The first and most straightforwards approach to learning hockey stops.
Begin with a carve, and apply more force to the edges closest to the skater's body such that you come to a stop during the movement.   

The focus should be on the sharpness of the carve at low speeds, while applying force on the frontward facing outside edges on both skates, and leaning backwards into the stop. 

Ideally the skater should end with their feet parallel.

{\bf Common Pitfall:} Skaters may fail to bring their back leg through the carve and end up in a power-slide instead of a hockey stop. 

{\color{red} TODO: Figure}


\subsection*{Developmental Drill: Question Marks} 

A developmental drill that helps with skaters that have the initial movement down, but have not yet got the stopping portion of the hockey stop nailed down.      

Place two cones 10ft apart. 
Starting from the rear cone, skate forwards and carve around the front of the front cone.       
If the skater stopped then they've managed to do a hockey stop, if they did not stop continue the carve through and around till the skater is facing the rear cone. 
Repeat the motion, increasing speed to help get a better edge for stopping.



\subsection*{Developmental Drill: Lateral Approach} 
% Roman Vitreol

Another approach to learning hockey stops is to begin with a lateral movement rather than a forwards one.    
The advantage of this is that there is less distance to turn, the hip movement is already incorporated, and having a set line to stop on may assist conceptually.   


Starting with a lateral, perform a C cut to stop. 


\subsection*{Developmental Drill: Straight On}
% Biffy Miffy

This is similar to the lateral hockey stops. 
Rolling at a low speed in a forwards direction the skater should attempt to shift their weight to their front wheels and push out with their rear wheels, turning to enter a stopping position. 

While moving to this position they should begin to engage the stop.
The skater's weight should shift to the edges of their skates closest to their body, with weight loaded mostly on the leg closest to their body. 

{\color{red} TODO: Figure}
