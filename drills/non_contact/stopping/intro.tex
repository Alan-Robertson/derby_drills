\chapter{Stopping}
\label{ch:stopping}



Most warm up stopping drills are agnostic to the style of stop being used. 
As a result we provide those drills here, they may be applied to any of the stops in this section.  

 

\subsection*{Drill: Jackie Daniels} 
\label{drill:stopping/jackie_daniels}


The classic stopping drill.
Skaters begin approximately 20 ft apart facing each other (the width of the infield of a derby track).

For a period of time (i.e. 45 seconds, 2 minutes etc) one skater skates across the track, and attempts to perform the stop as close to the line as possible without going over, before skating back and repeating the stop on the other line.   

Once the time is up the skater rests while the skater they were facing performs the drill. 

Attention should be paid to performing the stop on both sides.  


The focus in this drill should be firstly on decreasing stopping distance, then increasing speed while maintaining that stopping distance.

{\bf Progressions:}

A simple progression is to skate backwards out of the stop, and perform an equivalent backwards stop on the other line.
Plough stops should be paired with backwards plough stops. One footed ploughs with power slides, T-stops with backwards T-stops and hockey stops with backwards hockey stops.


\subsection*{Drill: Circular Stopping}
\label{drill:stopping/circular}

This is a similar generic stopping drill.  
Unlike Jackie Daniels which allows skaters to work at their own pace, this variant begins to add some notion of pace.


Skaters space themselves in pairs in 20ft increments around the track. 
On a whistle one skater in each pair is to accelerate and stop at the next 20ft marker. 
On the subsequent whistle the other member of the pair is to do the same. 

Whistle timing may be varied, at the start of the drill skaters may be allotted 5 seconds per stop, towards the end this can be brought down to 3 seconds. 
