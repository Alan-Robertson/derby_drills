\section{One Footed Plough Stops}
\label{sec:stopping/plough_one_foot}

Application depends on the floor.

Developing skaters should lock in one leg and plough with the other. 
Weight should be distributed onto the front inside wheel, while the back wheels are free to pivot, controlling the stop. 

Progression is to moving the stopping leg forward and only turning the ankle rather than any sweeping motions. 


\begin{figure}
\centering
\includegraphics[scale=0.25]{img/one_foot_plough/one_foot_plough}
\caption{Tight one foot plough.}
\end{figure}



\subsection{Developmental Drill: Non-passive Ploughs}
\label{sec:stopping/plough_one_foot/cuts}

In this drill we begin to replace the `eggshell' model of plough stops with a C cut model. 

While at rest start with both feet parallel. 

Move one foot forwards with weight distributed to the inside front wheel, while pushing out with both back wheels.  
This forms the basis of a plough stop.
