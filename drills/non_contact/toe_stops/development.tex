\section{Development}
\label{sec:toe_stop/development}

This section contains a collection of conditionning drills for toe-stops. 
These are suitable for skaters that are becoming comfortable with their toe-stops, but cease being useful after that.   


\subsection*{Developmental Drill: Balances}
\label{drill:toe_stop/balances}

While stationary, the skater should start on skates adopt a staggered stance, and pop up onto their toe stops. 

{\it Progressions:}
One foot balances.
Skaters should already be comfortable balancing on one foot before they attempt this.
The main complexity here is the loss of the staggered stance, skaters should be low with knees bent while balancing.


\subsection*{Developmental Drill: Popping Up}
\label{drill:toe_stop/popping_up}
The skater begins by skating backwards.
They should then stagger their stance and lean onto their toe stops, coming to a stop. 

After this, they should resume skating backwards and repeat.


{\it Progressions:}
The first progression of this skill is for the sktater to not come to a stop, but instead burn off speed, then drop back down onto their wheels and continue skating.


Another progression is to do this on one foot, skate backwards, lift a foot, drop onto the toe stop either to stop or slow. 
If slowing on one foot the skater should be encouraged to drop back to that one foot rather than two feet immediately.   



\subsection*{Developmental Drill: Weight shifting}
\label{drill:toe_stop/weight_shift}
The skater should begin stationary in a flat stance with skates facing slightly inwards.
Start by popping up onto toe stops from this non-staggered stance.

Next shift weight to one side and lift the other foot, then place that foot, shift weight and lift the first foot. 
When placing feet it is essential that the placement is of the toe-stop first, then the front two wheels, otherwise a developing skater is liable to lose their balance when trying to support themselves with just the front two wheels.


{\it Progressions:}
Eliminate the placement step so that it becomes a hop from toe-stop to toe-stop.  

Increase the distance between the legs that weight is being shifted over. 
Increase the pace of the shift.
With high enough pace and far enough distance this is no longer a developmental drill, but juking practice.
