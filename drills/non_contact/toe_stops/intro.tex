\chapter{Toe Stops}
\label{ch:toe_stops}

This chapter deals with basic toe stop movement techniques.


We will begin with some basic toe-stop guidance.
Our primary concern with toe-stop movement is stability.
There is an inherent risk of injury when falling or twisting a toe-stop.  
For this reason the first thing to check is that the toe stops are securely mounted (i.e. not turning).   
Next, it is imperative that skaters do not attempt manoeuvres which they do not have the balance and strength for.
Slow but steady development is key here. 


To core idea of the toe-stop is the `front tripod' formed by the toe-stop and the front two wheels.  
This needs to be placed both carefully and firmly enough that the wheels bind rather than roll, otherwise the skater will fall backwards. 
When practicing, distributing weight to the toes and leaning slightly forwards is strongly encouraged for two reasons.
\begin{itemize}
\item With a weight distribution on the front more weight is placed into the toe-stop rather than the wheels in the tripod.   
\item In the worst case falling forwards onto a knee is infinitely prefereable to falling backwards when wheels slip out from under toe-stops. 
\end{itemize}



