\chapter{Transitions}

This section details a collection of differing transition techniques.
As different skaters will find different approaches here easier or harder, if you are learning then some degree of trialing techniques to find which ones work for you is a good idea. 

These vary in difficulty, and it is advised to try thhe easier ones before progressing to the harder ones.
You should consider that most of the one footed transitions are inappropriate for a beginner skater.\\


Before attempting transitions you should have a degree of comfort with both one footed glides and backwards skating.  

If you are a beginner, the ones you may wish to consider include:
\begin{itemize}
\item Turnout - traditional `textbook' transition style
\item Staggered - easier than the turnout, good style for refereeing    
\item Pivot - an approach where your feet don't leave the ground. Some practice with manuals may help here.  
\item Partial jump - an approach that adds a small jump to assist with the turn. Good for starting but discouraged for derby gameplay due to instability.   
\end{itemize}

Conversely, more experienced skaters are encouraged to learn as many of these as possible, purely for technique practice.  

As part of all transitions practice, one foot glides and side surfing will help with balance, stability on one foot through the transition and conditioning the ligaments in the hips to open up to wider angles (depending on the transition technique).




It should first be noted here that basic refers to the relative difficulty of these transitions, not that they should be superceded by more complex transitions.

Tight, strong transitions that withstand impact from other skaters are almost always preferable to one footed jumps that leave the skater vulnerable to a hit.  


There are a few general conceits that should be discussed with these introductory transitions. The first is that practicing these while stationary is strongly preferred before attempting them while moving. 



The mechanism of turning will vary between different forms of transition, so that particular discussion will be deferred to the relevant sections.   
What is universal to all forms of transition is the difference in weight distribution in the feet as the skater's relative motion goes from forwards to backwards (or vice versa).   
This is less important for transition styles where both feet flip simultaneously.       


When turning from forwards to backwards, the skater's weight will begin in the heels of their feet during their forwards motion. 
Depending on the position of the skater and their relative balance and weight distribution, after lifting the first foot, moving the weight of the remaining foot to the heel (as when skating backwards) will prevent the forward foot from `running away'. It will also slow the skater's motion slightly, leading to a more controlled turn.     


Similarly when turning from backwards to forwards, initially the weight is in the heels. On the trailing foot this should be shifted to the toes. 


\subsection*{Generic Transition Warm-Up Drills}
There are a few generic transition warm-up drills, these apply to all types of transitions.

Additionally transitions are incorporated in turn around toe-stops style stops.

{\bf Transition on the Whistle}
Perform a






