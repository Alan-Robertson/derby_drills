\section{Turnout Transitions}
\label{sec:transitions/turnout}

The `classic' transition.

\begin{itemize}
\item Begin with feet parallel and facing forwards. 
\item Turn head and shoulders from the waist either left or right as far as it comfortable.   
\item Lift the leg corresponding with the direction you have turned transferring your weight to the other foot, and open up at the hips to as close to 180 degrees as possible.     
\item Place that foot, leaving you in an open side surf position.  
\item Transfer weight to the back foot, raise the other leg and complete your turn.   
\item Place your second foot parallel to the first. 
\end{itemize}

This can be practiced on the spot, or on a line while stationary. 

Due to the potential lack of elasticity in the ligaments of the hip, it is unlikely that a skater will achieve a full 180 degrees without prior conditioning.

This additionally leads to problems when performing this style of transition in motion - if the skater cannot open up their hips wide enough then they may end in a situation where their second foot is perpendicular to the first.

To ameliorate this, any angle beyond 90 degrees can be used if the skater allows a change in their trajectory to the new direction of their skate.
By then correcting their trajectory once they have raised their other foot the skater can still transition effectively.   
The greater their initial speed, the greater the strain on their edges when attempting this change in trajectory. 
For this reason this transition is more effective at speed as you get closer to 180 degrees.
