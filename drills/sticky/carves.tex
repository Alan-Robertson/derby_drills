Basic Carving


Carving is where the skater's skates are nearly colinear with the weight distributed either over the front leg, or back leg.  


When turning on the side of the front leg the skater should lean into the turn.  
When turning on the side of the read leg the skater should pre-emptively bring the rear leg forward through the turn and switch leading legs as they lean into the turn.  
When turning the skater's weight should be on the edge closest to their body.   

A common warm-up drill is to carve within lanes or from line to line in derby direction, alternating leading leg with each turn.    

Emphasis should be on the maintaining form and the smoothness, sharpness of the turn. 

\begin{itemize}
\item $n$ laps carving, 1 lane 
\item $n$ laps carving, 2 lanes
\item $n$ laps carving, line to line (4 lanes)
\end{itemize}


Progressions:
Progressing with carving is a matter of increasing speed and taking sharper angles without hockey stopping.   

Another progression is to move enough weight to the inside edge that only those wheels make contact with the ground.     

Unlike the earlier progression, another approach is to replace the turn of the carve with a hockey stop on the line using the same motion.  

Common pitfalls:

Developing skaters may be concerned about their loss of momentum on the turn. 
Momentum in the new direction is gained by the cutting motions in the turn, and dropping weight into the turn.   
Both of these will develop with time as the skater's sticky skating improves and should not be a cause for concern.


Not pulling the trailing leg through the turn early enough may lead to the skater exhibiting a loss of balance and a highly oblique angle.   

