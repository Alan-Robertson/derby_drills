\documentclass{journal}

\usepackage{amsmath}
\usepackage[nopatch]{microtype}
\usepackage{booktabs}
\usepackage{hyperref}


\usepackage[left=2.45cm,top=3cm,right=2.45cm,bottom=2.5cm,bindingoffset=0cm]{geometry}

\title{Receiving Impact}

\author{}

\begin{document}
\maketitle
\noindent 

This session is a developmental session focused on delivering and receiving impacts in a safe manner (i.e. derby stance).
An additional focus is on improving basic derby skills such as laterals and transitions. 
The session is targeted more for the development of newer skaters, with progressions for more experienced skaters.        


As braced two walls and three walls provide additional stability that is often a crutch for newer skaters, these drills focus on removing this support. 


\section*{Overview}

{\bf Training Preamble}
\begin{itemize}
\item Warm-up, forwards and backwards with \hyperref[drill:sticky:sticky_one_foot]{variant sticky skating} and \hyperref[drill:one_foot:serpentine]{one foot drills}, intersperse with fast lap warm up `pyramids' - 10 Minutes
\item Falling
\item One foot ploughs - 10 minutes
\item \hyperref[drill:laterals]{Laterals with variants} - 10 minutes 
\item \hyperref[drill:laterals:mirrored]{Mirrored laterals} - 10 Minutes 
\end{itemize}

\vspace{1cm}

{\bf One on One Blocking}
\begin{itemize}
\item Jammer Driving
\item Jammer Driving, Jammer disengages 
\item Jammer Driving, blocker disengages
\item Jammer Hits
\item Blocker Driving
\item Blocker Driving, Jammer disengages
\end{itemize}


\vspace{1cm}
\noindent 
{\bf Flat two walls}


\pagebreak
\section*{Warm-up}
\label{sec:warmup}

\subsection*{Sticky Skating [Variant]}
\label{drill:sticky:sticky_one_foot}

Developing skaters may be compensating for not fully developed edge-work in sticky skating. Breaking down the motions and forcing each to pull their own weight should show this up quite quickly.  
Repeat the following skating forwards and backwards. 
\begin{itemize}
    \item 1$\times$ lap right leg pushing out, left leg stationary  
    \item 1$\times$ lap left leg pushing out, right leg stationary 
    \item 1$\times$ lap right leg pulling in, left leg stationary 
    \item 1$\times$ lap right leg pulling in, right leg stationary 
\end{itemize}

To put the motions back together:
\begin{itemize}
    \item 2$\times$ laps regular sticky skating. 
\end{itemize}


\subsection*{Carves and Weaves}
\label{drill:sticky:carves_and_weaves}
Weaving should focus on sharp hip movements, keeping feet close together and sharpness of edges. More experienced skaters are encouraged to try to maintain an edge on the inside two wheels of each skate.     
Carving should focus on the sharpness of the carve.
For more experienced skaters line to line carves should aim to incorporate a secondary motion to maintain momentum. For backwards carves this would ideally be a pull under.    

As before this should be repeated skating forwards and backwards
\begin{itemize}
    \item $\frac{1}{2}\times$ lap weaving, 1 lane 
    \item $\frac{1}{2}\times$ lap weaving, 2 lanes
    \item $\frac{1}{2}\times$ lap carving, 2 lanes 
    \item $\frac{1}{2}\times$ lap carving, line to line
\end{itemize}



\section*{Stopping Drill}
\label{drill:stopping:jackie_daniels}
Basic Jackie Daniel's drill.

More experienced skaters should focus on decreasing stopping distance, then on increasing speed

\begin{itemize}
    \item Plough stops - More experienced skaters should focus on one footed plough stops, emphasis should be on form, then stopping distance, then speed 
    \item T stops (may be excised for time constraints) - experienced skaters should focus on T stop variants (out to the side, and out in front)  
    \item Turn around toe stops - More experienced skaters should focus on varying their transitions (tight, opening and closing hips, serial jump, two foot jump, one foot transitions)    
    \item Power-slides - Less experienced skaters may substitute with plough stops
    \item Hockey stops - Less experienced skaters may substitute with plough stops
\end{itemize}

% Falling Drills

\section*{Laterals}
\subsection*{Developmental Drill: Lateral Balancing}
\label{drill:laterals/developmental_balancing}

The first issue that new skaters may encounter is that while they may be proficient in one footed glides with their legs parallel, they may not have practiced with their legs akimbo.     
As this utilises different sets of muscles this may require some initial practice.


Begin by standing and balancing on one foot facing forwards with the foot turned out 90 degrees.
The other leg should be raised but still facing forwards. 
The skater's hips should be turned towards the turned out foot. 
Practice this on both legs.

\input{../../../../drills/non_contact/laterals/developmental/glide}
\input{../../../../drills/non_contact/laterals/developmental/static_weight_transfer}
\input{../../../../drills/non_contact/laterals/developmental/rolling_weight_transfer}
\subsection*{Drill: Mirrored Laterals}
\label{drill:laterals/mirrored}

Skaters work in pairs, approximately 5ft apart. 
One skater is leading, the other follows.
Skaters perform laterals across the track, with the `blocking' skater tracking and following.

This may be performed both with skaters facing each other, or the tracking skater facing away from the skater they are tracking. 



{\it Variation}
For a larger group.
Skaters should all face in the same direction around the track, each skater tracks the skater two in front of them.
Alternatively, if doing the backwards variant then they should track the skater two behind them.



\section*{One on One Blocking}
\subsection*{Developmental Drill: Driving}
\label{drill:one_on_one/developmental/driving}

Single jammer and blocker, the jammer begins behind the blocker and engages, driving them forwards.
Alternate sides.


The particular concern of this drill is that the jammer drives using their toe-stops and that they maintain their feet underneath them.  
Jammers should seek to push upwards as they drive rather than forwards.


For this drill blockers should maintain a derby stance, and look to keep their centre of gravity lower than the jammer's.
As a development drill, blockers should not yet worry about stop blocks.

\subsection*{Developmental Drill: Jammer Disengage}
\label{drill:one_on_one/developmental/jammer_disengage}

Single jammer and blocker, the jammer begins behind the blocker and engages, driving them forwards.
The jammer will sporadically pull back, testing if the blocker is maintaining their own balance rather than leaning back into the jammer. 
Alternate sides.


\subsection*{Developmental Drill: Jammer Juggling}
\label{drill:one_on_one/developmental/jammer_juggling}

Single jammer and blocker, the jammer begins behind the blocker disengaged. 
The jammer should sporadically hit and bounce the blocker forwards, the blocker should receive each impact, be driven forwards legally and maintain balance. 


Alternate sides.


\subsection*{Developmental Drill: Blocker Disengage}
\label{drill:one_on_one/developmental/blocker_disengage}

Single jammer and blocker, the jammer begins behind the blocker and engages, driving them forwards.
The blocker will sporadically move forwards, representing the blocker maintaining pack position. The jammer should always be ain a position that they do not fall forwards as this occurs.

Alternate sides.


\subsection*{Developmental Drill: Lateral Driving}
\label{drill:one_on_one/developmental/lateral_driving}

Single jammer and blocker, the jammer begins by driving forwards and towards the centre of the track. 
The blocker should perform a lateral movement, driving the jammer towards a line while also moving slightly forwards to avoid a direction of gameplay penalty.

Once the jammer is taken to a line the jammer should switch sides, driving back towards the middle of the track.
The blocker should switch to their other lateral movement, and drive the jammer to the other line. 
 

{\it Progressions:}
One extension of this drill is to allow the jammer the ability to change direction at any time.
The blocker should prepare for the disengage by stopping, and recapturing the jammer.

\subsection*{Developmental Drill: Lateral Capture}
\label{drill:one_on_one/developmental/lateral_capture}

This drill carries on from the lateral driving drill.


Single jammer and blocker, the jammer begins behind and aiming to move past the blocker with a small roll.
The blocker should perform a lateral into the jammer, capturing them and moving into a drive towards the line. 








\section*{Flat Two Walls}

\subsection*{Look Mum, No Hands}
\label{drill:three_wall/no_hands}

Alternatively known as `Look Ma, No Hands' depending on which side of the english language you reside.

This drill is a basic three wall with one jammer drill except the brace is not permitted to support the butts. 
The brace should drift in what would be a useful position providing communications and at best emotional rather than physical support.  

The drill should begin with the jammer only driving at the seam, with the brace tracking the position of the jammer.


{\it Progressions}
Rather than just driving the seam, the jammer is now free to juke and take lines. 
Unlike a regular two wall drill, the addition of a brace for communications means that the other blockers can hold a better posture. 




\section*{Two Wall Laterals}
\label{drill:two_wall:laterals}

Flat two walls, line to line laterals.  
More experienced skaters should use toe stop starts and push their fellow blockers.  

\section*{Two Wall Tracking}
\label{drill:two_wall:tracking}
In groups of three. 
Two blockers, one jammer.
Jammer works on moving between lanes 1 and 4, while the two wall works on catching and holding on the lines.
Emphasis on holding and not hitting out.      
Two walls should be flat.

Progression is to include jammer suck-backs.

\section*{Two Wall Driving}
\label{drill:two_wall:driving}
In groups of three. 
Two blockers, one jammer.
Jammer works on moving between lanes 1 and 4, while the two wall tries to drive the jammer.  
Focus should be on holding the jammer using the ribs and driving to the line, but not hitting out. 
Two walls should be flat.
As a progression the jammer should challenge the drive, and occasionally disengage for a suck back.

If training a dedicated Jammer, then keep the jammer in position, else rotate.
Blockers should rotate between inside and outside. 


\end{document}
