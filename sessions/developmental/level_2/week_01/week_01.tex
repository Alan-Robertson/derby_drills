\documentclass{journal}

\usepackage{amsmath}
\usepackage[nopatch]{microtype}
\usepackage{booktabs}
\usepackage{hyperref}


\usepackage[left=2.45cm,top=3cm,right=2.45cm,bottom=2.5cm,bindingoffset=0cm]{geometry}

\title{Week 1}

\author{}

\begin{document}
\maketitle
\noindent 

This is mostly a revision session to warm up to level 2 after the end of level 1.

\section*{Overview}

{\bf Training Preamble}
This is the first level 2 session. The goal is mostly a warm up and return to skating, with a secondary goal of spending some time working on the more difficult level 1 skills.

\subsection*{Warm-Up - 10 Mins}


\vspace{0.25cm}

\begin{itemize}
    \item Sticky skating - Regular, large and figure 8 eggshells
\end{itemize}

\vspace{0.25cm}


\subsection*{Revision - 20 mins}

Rather than part of the warm up, this is a chance to revisit these skills.

\vspace{0.25cm}
{\bf Weaves}
Start with two lane, focus is on maintaining edges and not lifting feet.


\vspace{0.25cm}
{\bf Carves}
As with weaves, the focus is on clean edge work, weighting and position.



\subsection*{Falls - 5 Mins}
This is a good time to revisit falling.

Run as a two sides of the track drill - one side with one fall, the other with the other.
Don't run this drill for too long, especially if skaters have poor kneepads.
\begin{itemize}
    \item \hyperref[drill:falls/knee_taps]{Knee taps} and \hyperref[drill:falls/double_knee]{Double knee taps}
    \item \hyperref[drill:fall/turtle]{Turtle falls} and \hyperref[drill:falls/slide]{Slide falls}
\end{itemize}



\subsection*{Stops - 10 mins}

Some basic jackie daniels stopping drills.
Mostly just warming up after the break.

\begin{itemize}
    \item Plough stops, one foot if applicable. 
    \item T Stops
\end{itemize}


\subsection*{Balancing Stepping and Walking - 10 mins}
Getting ankles engaged.

\begin{itemize}
    \item One foot balancing
    \item One foot balancing, leg turned out
    \item Stepping
    \item Toe stop balancing 
    \item Toe stop weight transfer
    \item Toe stop walking  
\end{itemize}

\subsection*{Backwards Skating - 10 Mins}
A few laps of backwards skating, either sticky or backwards duck walk.
The goal here is to stretch out the legs after stepping and walking and as a preamble to transitions. 

\subsection*{One Foot Skating - 15 Mins}

Looking at either gliding or starting to integrate edge-work, depending on where the skater's development is at.

The first goal should be for skaters to complete at least three quarters of a lap on a single glide - ideally they should be able to reach the full lap. 
Attention should be paid to balance as their momenum tails off.

For skaters who can comfortably perform this glide the focus can either be on \hyperref[drill:one_foot/circles]{single foot circles} or \hyperref[drill:one_foot/serpentine]{serpentine}.


\subsection*{Sideways Glides - 10 mins}

Gliding while holding a sideways posture, shoulders turned inwards or outwards rather than in the direction of motion. 

This is a prelude to laterals

\subsection*{T Stop Laterals - 10 mins}

An initial lateral motion, currently just putting a push, a sideways glide and then ending with a T stop.
The focus is to work on the balance required with the lateral motion rather than introducing the weight transfer and stopping components. 


\subsection*{Transitions - 10 mins}
Revise transitions.
It's best to take this section on a skater by skater basis, a selection of drills is \hyperref[[drill:transitions/staggered]{provided}.

The goal is to move from transitions to turn around toe-stops.


\newpage
\section*{Drills}

\subsection*{Knee Taps}
\label{drill:falls/knee_taps}

A controlled knee fall.
Skaters should lower themselves while skating and tap a knee pad to the floor regain control, then rise again to a skating position.  


\section{Knee Falls}
\label{sec:falls/double_knee}

A useful skill for falling.

The goal of this skill is to safely fall on two knees. 

As with the single knee fall, rather than falling directly down, the primary goal should be to burn off speed at an oblique angle. 
The shallower the angle the lessen the impact. 

Unlike the single knee fall we now need to content with two knees falling.
In this situation the goal is to stagger the timing of the impact of the knees. 

As there is a risk of tipping forwards it is advised to lean backwards from the hips in an ersatz rock-star pose. 



\subsection*{Turtle Falls}
\label{drill:fall/turtle}

While performing double knee falls there was a concern about pitching forwards.
This fall is about managing that situation. 


Here the skater begins with a double knee fall, they then pitch forward making contact with the ground with both their wrist and elbow pads.   
The concern here is how to avoid injury from this position.
The advice is to `turtle' - to assume as small a pose as possible by retracting arms and bringing the head in, while keeping the fingers either clenched or folded out of the way.    

The skater should then re-emerge from their shell once it is safe to do so.  

\section*{Slide Falls}
\label{drill:falls/slide}

So far the falling drills have assumed that you are falling forwards.
This technique exists for situations where that is not the case.

Our primary goal is to avoid falling either backwards, or directly onto a hip.  
The idea is then to `pick a cheek', to fall onto.  


The fall them maximises contact area along the glutes and thigh, and as with the other falls aims to turn it into a slide.

Falling directly onto your backside may result in a tailbone injury, which is ill advised.  


\subsection*{Drill One Foot Circles}
\label{drill:one_foot/circles}
A set of four movements (eight counting backwards) for developing edgework. 


The skater should find an open space on the track, and attempt to hold an edge on one foot, either inside or outside to start curving their skating.   
No active movement of the foot should be required to achieve this - if the skater is lifting or adjusting their wheels then they're forcing the turn rather than using their edges. 


\subsection*{Warm Up Drill: Serpentine} 

As different extension to one footed weaves.

Rather than two lanes of the track at a sharper angle than 45 degrees glide from line to line on one foot. At the line, or as close to it as the skater feels comfortable use the edges of the skate to turn, as in a weave or a carve to face the other line and continue the motion.   

As we're not currently generating momentum from this motion after two or three turns change feet. 
The focus is on the sharpness of the turn, rather than the speed. 
Speed may be added after technique has been polished.


\section{Staggered Transitions}
\label{sec:transitions/staggered}


The staggered transition is similar to the turn out, but rather than starting with feet parallel we begin with them staggered.  
This stance begins with the skater's hips already open, which removes that portion of the manoeuvre from the turn out transition.  

\subsection*{Developmental Drill: On the Spot} 

With or without skates, adopt a split stance. 
The skater should flip their back leg, then their front leg to face in the opposite direction.  


\end{document}
