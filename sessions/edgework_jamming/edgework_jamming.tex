\documentclass{journal}

\usepackage{graphicx}
\graphicspath{{../../drills/img}}


\usepackage{amsmath}
\usepackage[nopatch]{microtype}
\usepackage{booktabs}
\usepackage{hyperref}

\usepackage[left=2.45cm,top=3cm,right=2.45cm,bottom=2.5cm,bindingoffset=0cm]{geometry}

\title{Edgework Jamming}

\author{}

\begin{document}
\maketitle
\noindent 

This session is focused on jamming techniques using edgework.
The primary goal of these techniques is to leverage the blocker's momentum using faster turning rates to leave them unable to zone the jammer.     

These techniques work best if the jammer or the wall is in motion.


\section*{Overview}

\subsection*{Training Preamble - 15 Minutes}
\begin{itemize}
\item Warm-up, forwards and backwards with \hyperref[drill:sticky:eggshells]{Eggshells}, \hyperref[drill:one_foot:serpentine]{one foot drills} and  \hyperref[drill:carves_and_weaves]{edgework} - 5 Minutes
\item Sideways Leaps
\end{itemize}


\subsection*{Edgework Skills - 45 Minutes}
\begin{itemize}
    \item \hyperref[drill:edges:stationary_weaves]{Stationary Backwards Weaves} - 5 minutes
    \item \hyperref[drill:edges:stationary_carves]{Stationary Backwards Carves} - 5 minutes
    \item \hyperref[drill:edges:c_cut]{C-Cuts} - 5 minutes
    \item \hyperref[drill:edges:drop_step]{Drop Step Laterals} - 5 minutes
    \item \hyperref[drill:edges:c_cut_drop_step]{C-Cut Drop Step} - 5 minutes
    \item \hyperref[drill:edges:u_cut]{U-Cuts} - 5 minutes
    \item \hyperref[drill:edges:drop_step_u_cut]{Drop Step U-Cut Spin} - 5 minutes
    \item \hyperref[drill:edges:slow_one_foot_hockey]{Slow One Foot Hockey} - 5 minutes
\end{itemize}

\subsection*{Line to Line Jamming - 45 Minutes} 
\begin{itemize}
    \item \hyperref[drill:line_to_line_jamming:cutting]{Run the opposite line} - 10 minutes
    \item \hyperref[drill:line_to_line_jamming:juke_approach]{Juke Back} - 10 minutes
    \item \hyperref[drill:line_to_line_jamming:iterate]{Disengage and Iterate} - 10 minutes
    \item \hyperref[drill:line_to_line_jamming:Engage Juke]{Drop-Step Juke} - 10 minutes
    \item \hyperref[drill:line_to_line_jamming:bean_dip]{Bean Dip} - 5 minutes
    \item \hyperref[drill:line_to_line_jamming:toe_stop_line_run]{Toe Stop Line Run} - 5 minutes
\end{itemize}

Time permitting - engaged with the three wall drills.

\pagebreak
\section*{Warm-up}
\label{sec:warmup}

\subsection*{Sticky Skating}
\label{drill:sticky:eggshells}

Regular sticky skating
\begin{itemize}
\item Eggshells 
\item Figure 8 Eggshells
\item Wide Eggshells
\item Small and Fast
\end{itemize}
Forwards and backwards

\subsection*{Carves and Weaves [Variant]}
\label{drill:sticky:carves_and_weaves}
Weaving should focus on sharp hip movements, keeping feet close together and sharpness of edges. More experienced skaters are encouraged to try to maintain an edge on the inside two wheels of each skate.     
Carving should focus on the sharpness of the carve while maintaining speed.
This includes a variant carving warm-up. 
Skaters should perform a cross-over or backwards cross over between each carve.
The goal should be to establish a good carving form at a given speed, then increase the speed until form is lost, work at that speed to re-establish form and repeat.

\begin{itemize}
    \item $\frac{1}{2}\times$ lap weaving, 1 lane 
    \item $\frac{1}{2}\times$ lap weaving, 2 lanes
    \item $\frac{1}{2}\times$ lap carving + Crossovers, 2 lanes 
    \item $\frac{1}{2}\times$ lap carving + Crossovers, line to line
\end{itemize}

As before this should be repeated skating forwards and backwards

\subsection*{Carves and Weaves [Variant]}
\label{drill:sticky:carves_and_weaves}
Weaving should focus on sharp hip movements, keeping feet close together and sharpness of edges. More experienced skaters are encouraged to try to maintain an edge on the inside two wheels of each skate.     
Carving should focus on the sharpness of the carve while maintaining speed.
This includes a variant carving warm-up. 
Skaters should perform a cross-over or backwards cross over between each carve.
The goal should be to establish a good carving form at a given speed, then increase the speed until form is lost, work at that speed to re-establish form and repeat.

\begin{itemize}
    \item $\frac{1}{2}\times$ lap weaving, 1 lane 
    \item $\frac{1}{2}\times$ lap weaving, 2 lanes
    \item $\frac{1}{2}\times$ lap carving + Crossovers, 2 lanes 
    \item $\frac{1}{2}\times$ lap carving + Crossovers, line to line
\end{itemize}

As before this should be repeated skating forwards and backwards



\subsection*{One Foot [Variant]}
\label{drill:one_foot:serpentine}
As a progression between gliding and self propelled one footed skating this warm up borrows from introductory artistic skating.
The core idea is rather than locking in the foot to a glide that developing skaters need to start using their edges on one foot to make the necessary turns.     

Skaters should pick up momentum and then skate from line to line in a serpentine motion on one foot. When they run out of momentum they may change feet. Doing this backwards may be challenging on the track so it is recommended that initially regular backwards one foot drills be used. 

Progression of this drill is in sharpening angles, maintaining momentum and incorporating self-propelled motions.  
\begin{itemize}
    \item $2\times$ laps serpentine one foot forwards 
    \item $\frac{1}{2}\times$ lap one foot glides/self propelled backwards 
\end{itemize}


\section*{Stopping Drill}
\label{drill:stopping:jackie_daniels}
Basic Jackie Daniel's drill.

More experienced skaters should focus on decreasing stopping distance, then on increasing speed

\begin{itemize}
    \item Plough stops - More experienced skaters should focus on one footed plough stops, emphasis should be on form, then stopping distance, then speed 
    \item T stops (may be excised for time constraints) - experienced skaters should focus on T stop variants (out to the side, and out in front)  
    \item Turn around toe stops [Variant] - Skaters should focus on eliminating slide with increasing speed. Skaters should also focus on lunging out of the stop.  
    \item Hockey stops - Less experienced skaters may substitute with more turn around toe stops 
\end{itemize}


\section*{Line to Line}
\subsection*{Cutting In}
\label{drill:line_to_line_jamming:cutting}

Starting from lanes 1.5 or 3.5 accellerate towards the opposite line.
If starting from stationary, the initial push may involve a toe stop, the rest of the movement must be on skates.
If starting with a roll then aim to start in a fowards direction and carve suddenly towards the opposite line.

Upon reaching the line carve back in. 
This path may be seen in figure \ref{fig:chr_1}.


The drill may be performed with cones or chairs representing blockers. 

\begin{figure}[h]
\begin{center}
\includegraphics[]{christmas_tree/chr_1}
\caption{Path of the carve.\label{fig:chr_1}}
\end{center}

\end{figure}


{\it Progression:} Include a pair of blockers who will track the incoming jammer. The blockers will attempt to remove the jammer from the track, while the jammer should aim to break through the seam, or end engaged pushing back towards the center.  


\subsection*{Juke Approach}
\label{drill:line_to_line_jamming:juke_approach}

Starting from lanes 1.5 or 3.5 accellerate towards the opposite line.
If starting from stationary, the initial push may involve a toe stop, the rest of the movement must be on skates.
If starting with a roll then aim to start in a fowards direction and carve suddenly towards the opposite line.

When approaching the wall, throw in a juke to move the wall away from the intended path.
Then either cut back towards the closer line, or continue towards the opposite line before carving in. 
This path may be seen in figure \ref{fig:chr_1}.

The drill may be performed with cones or chairs representing blockers. 

\begin{figure}[h]
\begin{center}
\includegraphics[]{christmas_tree/chr_2}
\caption{Path of both the carve and the possible juke.\label{fig:chr_1}}
\end{center}

\end{figure}


{\it Progression:} Include a pair of blockers who will track the incoming jammer. The blockers will attempt to remove the jammer from the track, while the jammer should aim to break through the seam, or end engaged pushing back towards the center.  

The jammer should seek to juke the blockers with the initial movement, or failing that should seek to challenge the blockers sufficiently in the rush to the line to pass them there.

For further progressions the jammer may incorporate additional jukes on the approach. 

\subsection*{Disengage and Iterate}
\label{drill:line_to_line_jamming:iterate}

This drill requires two to three blockers.

Starting from lanes 1.5 or 3.5 accellerate towards the opposite line.
Depending on the speed of the jammer, the blockers should start either in line with the jammer and track them, in the middle of the track and intercept them, or waiting on the opposite line for the catch.


If starting from stationary, the initial push may involve a toe stop, the rest of the movement must be on skates.
If starting with a roll then aim to start in a fowards direction and carve suddenly towards the opposite line.

Carve in from that line and engage with the wall.
The blockers should attempt to push the jammer out if they have any outwards momentum, otherwise their goal is to trap and pocket the jammer. 

The jammer should attempt to disengage their hips, and run back to the opposite line, attempting to outpace the blockers.
The jammer may also perform a juke and attempt to run the closer line, at the risk of being hit out by a blocker. 


When approaching the wall, throw in a juke to move the wall away from the intended path.
Then either cut back towards the closer line, or continue towards the opposite line before carving in. 
If carving in 

The drill may be performed with cones or chairs representing blockers. 

\subsection*{Cutting In}
\label{drill:line_to_line_jamming:bean_dip}

This drill requires a blocker

Starting from lanes 1.5 or 3.5 accellerate towards the opposite line.
If starting from stationary, the initial push may involve a toe stop, the rest of the movement must be on skates.

The blocker should accellerate towards the outside line, ready to engage with their shoulders, or backwards.
The jammer should attempt to turn upwards along the line and dip underneath the blocker and clear them completely.
The jammer may turn their shoulders slightly to present their back as an illegal target zone during this movement.

If the jammer feels that they cannot manage that they should attempt to juke back past the blocker's committed hit towards the center of the track.
The jammer should avoid back blocks from turning into the blocker, or low blocks from engaging too low.   
Blockers should avoid performing a high block or a back block on the jammer.
Dipping slightly early to give warning of the intended directory



%\begin{figure}[h]
%\begin{center}
%\includegraphics[]{christmas_tree/chr_1}
%\caption{Path of the carve.\label{fig:chr_1}}
%\end{center}
%
%\end{figure}


{\it Progression:} Incorporate an approach with a juke - the bean dip is the alternative for running the opposite line.   

\subsection*{Cutting In}
\label{drill:line_to_line_jamming:toe_stop_line_run}

This drill requires a blocker

Starting from lanes 1.5 or 3.5 accellerate towards the opposite line.
If starting from stationary, the initial push may involve a toe stop, the rest of the movement must be on skates.

The blocker should accellerate towards the outside line, engaging with their thigh, or backwards.

The jammer should attempt to perform a turn around toe-stop on the line, and then run before the blocker reaches.
If the blocker does reach then the jammer should be able to sneak past them.  


If the jammer feels that they cannot manage that they should attempt to juke back past the blocker's committed hit towards the center of the track.
The jammer should avoid back blocks from turning into the blocker, or low blocks from engaging too low.   
Blockers should avoid performing a high block or a back block on the jammer.
Dipping slightly early to give warning of the intended directory



%\begin{figure}[h]
%\begin{center}
%\includegraphics[]{christmas_tree/chr_1}
%\caption{Path of the carve.\label{fig:chr_1}}
%\end{center}
%
%\end{figure}


{\it Progression:} Incorporate an approach with a juke.









\end{document}
