\documentclass{journal}

\usepackage{graphicx}
\graphicspath{{../../drills}}


\usepackage{amsmath}
\usepackage[nopatch]{microtype}
\usepackage{booktabs}
\usepackage{hyperref}

\usepackage[left=2.45cm,top=3cm,right=2.45cm,bottom=2.5cm,bindingoffset=0cm]{geometry}

\title{Three Walls, Pockets and Shelving}

\author{}

\begin{document}
\maketitle
\noindent 

Focus of this session is about getting back to three walls.
Most of these drills are blocker focused, but jammers should be working on shelving.


\section*{Overview}

\subsection*{Off Skates Warm-up 5-10 Minutes Before the Session}
Lightly making sure that the joints still work.

Focus is {\bf not} on building strength, cardio or endurance, just on limbering up. Do 5 to 10 reps of each, on each side. 

\begin{itemize}
\item Toe taps + one leg light squats
\item Arm, shoulder and neck stretches
\item Opening and closing hip rotations
\item Knee raise to superman 
\item Cross chest knee raises
\item Butt kicks, high knees, side shuffles
\end{itemize}


\subsection*{Basic Warm-up - 15 minutes}

Mostly standard edge-work warm-up. Stage in faster laps rather than doing them all at once.
Do forwards and backwards.

\begin{itemize}
\item Sticky skating - Regular, Figure 8, low
\item Regular weaves + carves
\item One foot line-to-line (it not able, then \hyperref[drill:one_foot/serpentine]{serpentine}). 
\item One foot lateral 
\item Fast laps 
\item Stops
\end{itemize}


\subsection*{Edge-work - 20 minutes}
A bit of edge-work.

\begin{itemize}
    \item \hyperref[drill:one_foot/circles]{One foot circles}.
    \item \hyperref[drill:footwork/alternating_hockeys]{Drop step or Lateral stopping drill}. 
\end{itemize}

\subsection*{Contact Warm Up - 10 minutes}
\begin{itemize}
    \item Depending on numbers, either Spaghetti-meatballs or just one-on-one.
    \item Jammer and a three wall
\end{itemize}


\subsection*{Shelving, Driving and Taking Position - 20 minutes}
An extension of taking position from other skaters.

\begin{itemize}
\item Drive one skater to the line and out
\item Load drive one skater to pass them.
\item Drive a two wall 20 foot or until it breaks 
\item Drive a three wall 20 foot or until it breaks
\end{itemize}




\subsection*{Pocketing - 20 minutes}
Three wall drill
\begin{itemize}
    \item Jammers are looking to shelve and drive the wall 30 foot. 
    \item Blockers are attempting to pocket and drive the jammer to the line.
    \item If in the last 10 feet, controlled hit out on the jammer and reset. 
\end{itemize}

Numbers permitting, run two walls at once and use pack to determine what to do with the jammer, otherwise use a marker.




\subsection*{Four Walls}

{\bf Four wall drill}. 
The same as the above, with a solid three wall, and a fourth blocker. 

\begin{itemize}
    \item If the other wall is close, with no pack considerations, the fourth skater should pocket.
    \item The the other wall is ahead, then the fourth skater should comms and be ready to bridge. The wall then takes over pocket duties. 
    \item If the wall is at the front the fourth skater should comms from behind the wall, and be ready to bridge. The wall takes over pocket duties. 
\end{itemize}
Numbers permitting, run two walls at once and use pack to determine what to do with the jammer. Otherwise use a marker.




\subsection*{Finnish Pace Line}
And back to some more dynamic movement drills.
\begin{itemize}
    \item \hyperref[drill:footwork/finnish]{Finnish Snake drill}
\end{itemize}


\pagebreak


\section*{Drill Notes}

\subsection*{Warm Up Drill: Serpentine} 

As different extension to one footed weaves.

Rather than two lanes of the track at a sharper angle than 45 degrees glide from line to line on one foot. At the line, or as close to it as the skater feels comfortable use the edges of the skate to turn, as in a weave or a carve to face the other line and continue the motion.   

As we're not currently generating momentum from this motion after two or three turns change feet. 
The focus is on the sharpness of the turn, rather than the speed. 
Speed may be added after technique has been polished.

\section{Skating to Toe Stops and Back}
\label{sec:toe_stop/skating_to_toe}
% Pi

This section deals with alternating between skating, running on toe-stops and moving back to skating.

The primary difficulty with this skill is that toe-stops are typically angled such that jumping onto them while rolling without first matching momentum will result in the skater falling over.    
This angling also means that this skill is easier to practice backwards.


All of the following drills should be attempted at low speed first. 
If the skater is not comfortable moving at the required speed when starting on toe stops, they should not attempt to move onto toe stops from that speed when rolling.  

\subsection{Skating Backwards}
Skate backwards, hop onto toe stops for a couple of steps, resume skating backwards.
This is easier than the forwards version as the toe stops are already angled backwards. 

If the skater needs to bail out from this movement they can drop onto their toe stops and come to a stop. 


\subsection{Sideways}
Similarly to the backwards version, this can also be performed sideways while avoiding angling issues with toe-stops.  

From a side surf, incorporate a small jump landing on toe stops.
The skater then uses their existing momentum to perform a side shuffle or a grapevine on toe stops.   
They then return to a side surf.

If the skater is concerned by the sideways momentum on their toe-stops, the first toe-stop steps may be accompanied by a partial transition for greater stability.  


\subsection{Skating Forwards}
The hardest of the three. 
Care must be taken to place the toe-stops at the correct angle while moving forwards. 
Leaning forwards will help with this, as will maintaining momentum. 




\subsection*{Drill One Foot Circles}
\label{drill:one_foot/circles}
A set of four movements (eight counting backwards) for developing edgework. 


The skater should find an open space on the track, and attempt to hold an edge on one foot, either inside or outside to start curving their skating.   
No active movement of the foot should be required to achieve this - if the skater is lifting or adjusting their wheels then they're forcing the turn rather than using their edges. 




\subsection*{Drill: Alternating Hockey Stops}
\label{drill:footwork/alternating_hockeys}

This drill involves working on both forwards and backwards hockey stops.

If the skater is not comfortable with drop steps, they may be replaced with regular laterals.
If the skater is not comfortable with hockey stops, they may be replaced with powerslides.


\begin{itemize}
\item Start with both feet parallel and facing forwards.
\item Perform a drop-step into a lateral towards the right.
\item Perform a hockey stop, ending with feet parallel and facing the original direction.
\item Perform a drop-step into skating backwards towards the right.
\item Perform a backwards hockey stop, ending with feet parallel and facing the original direction. 
\item Repeat.
\end{itemize}

This should also be performed to the left.

{\it Progressions}
Replace the hockey stops with one foot hockey stops, only placing the other foot after coming to a complete stop.
Increase tempo and power.

% TODO Figure




\subsection*{Finnish Footwork Drill}
\label{drill:footwork/finnish}

This drill was poached from a Finnish ice hockey drill, hence the name. 

This drill may be run between the outside and inside lines of the track, but having some more room for a run-up is preferred (for example by running it across the inside of the track).

The skater should begin on the inside line while facing the outside line.
Throughout the drill their shoulders and torso should remain facing in this direction. 

\begin{itemize}
\item The skater begins by skating towards the outside line at a slightly oblique angle in derby direction. They may mix a crossover into this motion.
\item They should perform a carve, burning off their forward momentum while conserving their sideways momentum. 
\item From here the skater should perform backwards crossovers to propel themselves back towards the inside line.
\item Once at the inside line they should perform a backwards carve, resetting their position. 
\end{itemize}

As the sequence is repeated the skater will naturally move down the track.

The drill should be repeated in both directions.


\includegraphics[scale=0.25]{img/finnish_drill/path}


{\it Commentary}:
The drill has two main goals - keeping the skater's torso moving independently of their legs, and working on the blurred line between a hockey stop and a carve such that the skater can redirect their momentum efficiently.    


{\it Variant}:
The forwards and backwards carves may be reversed - the backwards carves being performed at the front line and the forwards carves at the back line.






\end{document}
