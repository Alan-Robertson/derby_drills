\documentclass{journal}

\usepackage{amsmath}
\usepackage[nopatch]{microtype}
\usepackage{booktabs}
\usepackage{hyperref}


\usepackage[left=2.45cm,top=3cm,right=2.45cm,bottom=2.5cm,bindingoffset=0cm]{geometry}

\title{Lateral Jammer Control}

\author{}

\begin{document}
\maketitle
\noindent 

The core goal of this session is to work on and develop lateral driving and tracking skills. Following the jammer leaves them in control of the wall, legally driving and challenging the jammer as a wall limits their options and lets the wall act rather than react.        

The session is targeted more for the development of newer skaters, with progressions for more experienced skaters.        

\section*{Overview}

{\bf Training Preamble}
\begin{itemize}
\item Warm-up, forwards and backwards with \hyperref[drill:sticky:sticky_one_foot]{variant sticky skating} and \hyperref[drill:one_foot:serpentine]{one foot drills} - 10 Minutes
\item \hyperref[drill:stopping:jackie_daniels]{Stopping drill}, with modification - 10 Minutes 
\item \hyperref[drill:laterals]{Laterals with variants} - 10 minutes 
\item \hyperref[drill:laterals:mirrored]{Mirrored laterals} - 10 Minutes 
\item \hyperref[drill:blocking:one_on_one]{One on one blocking} (Lateral Tracking) - 5 minutes
\end{itemize}

\vspace{1cm}
\noindent 
{\bf Tracking and Driving}
\begin{itemize}
\item \hyperref[drill:two_wall:laterals]{Two wall Laterals} - 5 Minutes 
\item \hyperref[drill:two_wall:tracking]{Two wall, line to line tracking} - 5 minutes 
\item \hyperref[drill:two_wall:driving]{Two wall, line to line driving} - 10 minutes 
\item \hyperref[drill:three_wall:driving_facilitated]{Three wall}, line to line drive and hold (Jammer facilitating) - 10 minutes 
\end{itemize}

\vspace{1cm}
\noindent
{\bf Derby Conditions}
\begin{itemize}
\item \hyperref[drill:three_wall:driving]{Three wall driving}, Jammer not facilitating - 10 minutes 
\item (Numbers permitting) \hyperref[drill:three_wall:proximity]{Two three walls in proximity} - 10 minutes  
\item (Numbers not permitting) \hyperref[drill:three_wall:offence]{Three wall with offence}, Jammer not facilitating - 10 minutes 
\end{itemize}

\pagebreak
\section*{Warm-up}
\label{sec:warmup}

\subsection*{Sticky Skating [Variant]}
\label{drill:sticky:sticky_one_foot}

Developing skaters may be compensating for not fully developed edge-work in sticky skating. Breaking down the motions and forcing each to pull their own weight should show this up quite quickly.  
Repeat the following skating forwards and backwards. 
\begin{itemize}
    \item 1$\times$ lap right leg pushing out, left leg stationary  
    \item 1$\times$ lap left leg pushing out, right leg stationary 
    \item 1$\times$ lap right leg pulling in, left leg stationary 
    \item 1$\times$ lap right leg pulling in, right leg stationary 
\end{itemize}

To put the motions back together:
\begin{itemize}
    \item 2$\times$ laps regular sticky skating. 
\end{itemize}


\subsection*{Carves and Weaves}
\label{drill:sticky:carves_and_weaves}
Weaving should focus on sharp hip movements, keeping feet close together and sharpness of edges. More experienced skaters are encouraged to try to maintain an edge on the inside two wheels of each skate.     
Carving should focus on the sharpness of the carve.
For more experienced skaters line to line carves should aim to incorporate a secondary motion to maintain momentum. For backwards carves this would ideally be a pull under.    

As before this should be repeated skating forwards and backwards
\begin{itemize}
    \item $\frac{1}{2}\times$ lap weaving, 1 lane 
    \item $\frac{1}{2}\times$ lap weaving, 2 lanes
    \item $\frac{1}{2}\times$ lap carving, 2 lanes 
    \item $\frac{1}{2}\times$ lap carving, line to line
\end{itemize}


\subsection*{One Foot [Variant]}
\label{drill:one_foot:serpentine}
As a progression between gliding and self propelled one footed skating this warm up borrows from introductory artistic skating.
The core idea is rather than locking in the foot to a glide that developing skaters need to start using their edges on one foot to make the necessary turns.     

Skaters should pick up momentum and then skate from line to line in a serpentine motion on one foot. When they run out of momentum they may change feet. Doing this backwards may be challenging on the track so it is recommended that initially regular backwards one foot drills be used. 

Progression of this drill is in sharpening angles, maintaining momentum and incorporating self-propelled motions.  
\begin{itemize}
    \item $2\times$ laps serpentine one foot forwards 
    \item $\frac{1}{2}\times$ lap one foot glides/self propelled backwards 
\end{itemize}

\subsection*{Hot Laps}

Emphasis is on track awareness, form and speed.
Depending on time constraints the backwards laps may be omitted.

\begin{itemize}
    \item $3\times$ laps crossovers 
    \item $3\times$ laps backwards crossovers 
\end{itemize}


\section*{Stopping Drill}
\label{drill:stopping:jackie_daniels}
Basic Jackie Daniel's drill.

More experienced skaters should focus on decreasing stopping distance, then on increasing speed

\begin{itemize}
    \item Plough stops - More experienced skaters should focus on one footed plough stops, emphasis should be on form, then stopping distance, then speed 
    \item T stops (may be excised for time constraints) - experienced skaters should focus on T stop variants (out to the side, and out in front)  
    \item Turn around toe stops - More experienced skaters should focus on varying their transitions (tight, opening and closing hips, serial jump, two foot jump, one foot transitions)    
    \item Power-slides - Less experienced skaters may substitute with plough stops
    \item Hockey stops - Less experienced skaters may substitute with plough stops
\end{itemize}



\section*{Laterals with Variants}
\label{drill:laterals}

As the focus of the session is on laterals and lateral jammer control we open with a brief laterals warm up. 
For the more experienced skaters variants are posed.

The goal should be for `skateboard push' style laterals from line to line.
Stops may either be stomps, C-cuts, T-stops, power slides or hockey stops. 
Upon stopping the skater's position should be on the line ready to take impact. 


Progression for more experienced skaters should be to incorporate other stopping styles and movement.

\begin{itemize}
    \item Stomp stops
    \item C-cut style stops 
    \item Back foot T-stop stops
    \item Hockey stops
    \item Toe-stop starts (pizza slices)
\end{itemize}


\section*{Mirrored Laterals}
\label{drill:laterals:mirrored}

Skaters should pair up both facing derby direction.
The back skater leads with laterals while the front skater should attempt to mirror. 

\section*{One on One Blocking}
\label{drill:blocking:one_on_one}
Focus of this drill is tracking. Jammers should not juke, but should work line to line and may also try to push past the blocker.

As the blockers know in advance that their jammer will not be juking they can focus on maintaining contact in motion.  


\section*{Two Wall Laterals}
\label{drill:two_wall:laterals}

Flat two walls, line to line laterals.  
More experienced skaters should use toe stop starts and push their fellow blockers.  

\section*{Two Wall Tracking}
\label{drill:two_wall:tracking}
In groups of three. 
Two blockers, one jammer.
Jammer works on moving between lanes 1 and 4, while the two wall works on catching and holding on the lines.
Emphasis on holding and not hitting out.      
Two walls should be flat.

Progression is to include jammer suck-backs.


\section*{Two Wall Driving}
\label{drill:two_wall:driving}
In groups of three. 
Two blockers, one jammer.
Jammer works on moving between lanes 1 and 4, while the two wall tries to drive the jammer.  
Focus should be on holding the jammer using the ribs and driving to the line, but not hitting out. 
Two walls should be flat.
As a progression the jammer should challenge the drive, and occasionally disengage for a suck back.

If training a dedicated Jammer, then keep the jammer in position, else rotate.
Blockers should rotate between inside and outside. 

\section*{Three Wall Driving (Jammer Facilitating)}
\label{drill:three_wall:driving_facilitated}

In groups of four. 
Three blockers, one jammer.
Jammer works on moving between lanes 1 and 4, while the three-wall tries to drive the jammer.
Walls should focus on not rotating and not hitting out.  
Brace should try to be colinear with the jammer. 
More experienced blockers may attempt to wrap the jammer.

\section*{Three Wall Driving (Jammer not Facilitating)}
\label{drill:three_wall:driving}

In groups of four. 
Three blockers, one jammer.
Jammer is free to jam.
Walls should focus on not rotating and not hitting out.  
Brace should try to be colinear with the jammer. 
More experienced blockers may attempt to wrap the jammer.
The jammer should challenge and attempt to escape.


\section*{Three walls in Proximity}
\label{drill:three_wall:proximity}

In groups of eight. 
Six blockers in two teams of three, two jammers.
Jammers are free to jam.
Walls should focus on not rotating and not hitting out.
Walls should also consider using the other wall to zone part of the track. 
Jammers should consider how to use their friendly wall to screen. 
Blockers should swap to Offence once they have lost the jammer.   

Wall Goals:
\begin{itemize}
\item Maintain pack
\item Drive the jammer to a line and hold them there.  
\item Recapture an escaping jammer and reform
\item Hit out an escaping jammer, reset and reform   
\end{itemize}


\section*{Three walls with Offence}
\label{drill:three_wall:offence}
In groups of five. 
Jammers are free to jam and challenge the wall.
More experienced skaters should perform offence as this session had no prior offence drills.  
Walls should focus on not rotating and not hitting out, while mitigating the impact of the offence.

Wall anti-offence goals:
\begin{itemize}
\item Avoid the offence 
\item Block of the offence 
\item Reform 
\end{itemize}


\end{document}
