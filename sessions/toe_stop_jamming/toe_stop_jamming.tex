\documentclass{journal}

\usepackage{amsmath}
\usepackage[nopatch]{microtype}
\usepackage{booktabs}
\usepackage{hyperref}

\usepackage[left=2.45cm,top=3cm,right=2.45cm,bottom=2.5cm,bindingoffset=0cm]{geometry}

\title{Toe Stop Jamming}

\author{}

\begin{document}
\maketitle
\noindent 

This training session is focused around an introduction to jamming ideas, and using toe stops, carves and simple spins.   

Core idea is for the jammer to not give their hips away and to selectively disengage them.

\section*{Overview}

\subsection*{Training Preamble} - 20-25 minutes
\begin{itemize}
\item Warm-up, forwards and backwards with \hyperref[drill:sticky:eggshells]{Eggshells}, \hyperref[drill:one_foot:serpentine]{one foot drills} and  \hyperref[drill:carves_and_weaves]{edgework} - 5 Minutes
\item Side Surfing
\item \hyperref[drill:stopping:jackie_daniels]{Stopping drill}, with modification - 5 Minutes 
\item Lateral Lunges - 5 Minutes 
\end{itemize}

% \item Sideways Leaps - For next time 

\subsection*{Toe Stop Warm Up} - 20 minutes
\begin{itemize}
    \item \hyperref[drill:toe_stop:loading]{Toe stop loading} - 2 minutes
    \item \hyperref[drill:toe_stop:stepping]{Toe stop stepping} - 2 minutes
    \item \hyperref[drill:toe_stop:side_shuffles]{Side Shuffles and Grapevines} - 2 minutes
    \item \hyperref[drill:side_rolling]{Toe Stops to Rolling and Back} - 15 minutes
\end{itemize}

\subsection*{Static Blocking} - 45 minutes
\begin{itemize}
    \item \hyperref[drill:static_block:lunge]{Lunging and Jabbing in Pairs.} Emphasise that when lunging to bring the other leg around. - 10 minutes 
    \item \hyperref[drill:static_block:line_shuffle]{Line Shuffle.} Solo drill with either chairs or cones. Work on angles. - 10 minutes
    \item \hyperref[drill:static_block:spin_to_win]{Spinning} Solo or in pairs. - 5 minutes solo, 10 minutes in pairs 
    \item \hyperref[drill:static_block:spin_lunge_feint]{Work against a three wall with spinning, lunging and feinting.} - 10 minutes
\end{itemize}

\subsection*{Active Blocking} - 20 minutes
\begin{itemize}
\item \hyperref[drill:active_blocking:drives]{Toe Stop Driving} - 10 minutes 
\item \hyperref[drill:active_blocking:rolling in]{Work on a three wall with 5-10ft of run up.} - Remainder of time 
\end{itemize}

\pagebreak
\section*{Warm-up}
\label{sec:warmup}

\subsection*{Sticky Skating}
\label{drill:sticky:eggshells}

Regular sticky skating
\begin{itemize}
\item Eggshells 
\item Figure 8 Eggshells
\item Wide Eggshells
\item Small and Fast
\end{itemize}
Forwards and backwards

\subsection*{Carves and Weaves [Variant]}
\label{drill:sticky:carves_and_weaves}
Weaving should focus on sharp hip movements, keeping feet close together and sharpness of edges. More experienced skaters are encouraged to try to maintain an edge on the inside two wheels of each skate.     
Carving should focus on the sharpness of the carve while maintaining speed.
This includes a variant carving warm-up. 
Skaters should perform a cross-over or backwards cross over between each carve.
The goal should be to establish a good carving form at a given speed, then increase the speed until form is lost, work at that speed to re-establish form and repeat.

\begin{itemize}
    \item $\frac{1}{2}\times$ lap weaving, 1 lane 
    \item $\frac{1}{2}\times$ lap weaving, 2 lanes
    \item $\frac{1}{2}\times$ lap carving + Crossovers, 2 lanes 
    \item $\frac{1}{2}\times$ lap carving + Crossovers, line to line
\end{itemize}

As before this should be repeated skating forwards and backwards


\subsection*{One Foot [Variant]}
\label{drill:one_foot:serpentine}
As a progression between gliding and self propelled one footed skating this warm up borrows from introductory artistic skating.
The core idea is rather than locking in the foot to a glide that developing skaters need to start using their edges on one foot to make the necessary turns.     

Skaters should pick up momentum and then skate from line to line in a serpentine motion on one foot. When they run out of momentum they may change feet. Doing this backwards may be challenging on the track so it is recommended that initially regular backwards one foot drills be used. 

Progression of this drill is in sharpening angles, maintaining momentum and incorporating self-propelled motions.  
\begin{itemize}
    \item $2\times$ laps serpentine one foot forwards 
    \item $\frac{1}{2}\times$ lap one foot glides/self propelled backwards 
\end{itemize}


\section*{Stopping Drill}
\label{drill:stopping:jackie_daniels}
Basic Jackie Daniel's drill.

More experienced skaters should focus on decreasing stopping distance, then on increasing speed

\begin{itemize}
    \item Plough stops - More experienced skaters should focus on one footed plough stops, emphasis should be on form, then stopping distance, then speed 
    \item T stops (may be excised for time constraints) - experienced skaters should focus on T stop variants (out to the side, and out in front)  
    \item Turn around toe stops [Variant] - Skaters should focus on eliminating slide with increasing speed. Skaters should also focus on lunging out of the stop.  
    \item Hockey stops - Less experienced skaters may substitute with more turn around toe stops 
\end{itemize}


\section*{Toe Stop Loading}
\label{drill:toe_stop:loading}

This drill is focused on being comfortable loading and moving between toe stops. 

Knees should be bent, weight shifted between planted legs, and toes should be pointed slightly inwards. 
Begin with staying on the spot and shifting weight between legs, starting to lift one leg, feel free to incorporate a small hop or jump into the motion.
The goal is to eventually be comfortable moving between toe stops. 


\section*{Toe Stop Stepping}
\label{drill:toe_stop_stepping}
Some basic toe stop stepping drills.

The next part resembles an aerobics class.
One skater leads while the others mirror their actions, they should step forwards, backwards, left and right.


\section*{Side Shuffles}
\label{drill:toe_stop:side_shuffles}

Extending on the hopping and jumping movements we now consider fast side shuffling and grape vines.  

Set up on track starting in lane 1 and perform a series of side shuffles 
\begin{enumerate}
\item Lane 1 to lane 2 and back to lane 1 
\item Lane 1 to lane 3 and back to lane 1
\item Lane 1 to lane 4 
\item Lane 4 to lane 3 and back to lane 4
\item Lane 4 to lane 2 and back to lane 4
\item Lane 4 to lane 1 
\end{enumerate}

Next, while moving, skaters begin in lane 2 or 3, and should leap on one foot over towards the opposite line, land, then leap back.   


\section*{Rolling to Toe Stops}
\label{drill:toe_stop:rolling}

This drill is about moving between skating and toe stops.
This should be taken only at a pace the skater is comfortable with.

Starting with skating backwards, begin stepping on toe stops backwards, then return to skating. 

Next, perform the same from a side-surf to a grape-vine or side shuffle. 
This should be tried either from a front or back loaded side surf.

Lastly skating forwards to toe stops to skating forwards. This will require leaning forwards. 

\section*{Lunging, Turning and Jabbing}
\label{drill:static_block:lunge}

In pairs one skater is the blocker, the other a jammer. 
The goal is to threaten either a lunge, a `jab' / feint, or turn across the back of the blocker by diengaging hips to force them to respond.

The jammer should bring their opposite leg over to threaten a lunge on the blocker.
The blocker needs to determine if this is a real threat, or whether it's merely a feint.
If the blocker doesn't respond then the jammer should be ready to convert their feint into a lunge. 

\section*{Line Shuffle}
\label{drill:static_block:line_shuffle}

The goal of this drill is to take a line as fast as possible.
This drill works best with two three walls in proximity.

Jammers should begin engaged in the seam of a two wall.

They should then attempt to identify a line that they wish to take, potentially using the other wall as a screen.

From here the idea is to run at the line, transition and toe-stop shuffle along the line. 

The transition into turn around toe-stop component depends on what feet are planted. 
If both feet are planted, then grape-vines are easier, if just the tailing foot is planted then a side-shuffle is easier.    
Planting the tailing foot requires a sharper stop. 

\section*{Spinning}
\label{drill:static_block:spin_to_win}
First practiced on the spot, then with a blocker. 

Plant a toe stop, then cross the opposite leg over, planting the other toe stop.
Spinning is assisted by disengaging the tripod that the skate is resting on, and instead balancing on just the toe-stop.
From this position it should not be clear if you are about to lunge, setting up the feint. 
Launch off the front toe stop, pivoting using the back one to reset your direction. 
Be ready from the new position to lunge.

Practice both a 180 spin and the full 360.

Then practice with a blocker.
With the blocker the goal is to wait for the blocker to make contact and begin to move the jammer before the jammer begins to spin.  
This ensures that the blocker has committed to the movement, and gives them a false victory when they first make contact. 

\section*{Spin Lunge Feint}
\label{drill:static_block:spin_lunge_feint}
Starting engaged with a three wall. 
Jammers should mix between spinning, lunging, feinting and line running to get past the three wall. 


\section*{Toe Stop Driving}
\label{drill:active_blocking:drive}
This is the back up plan, the idea is to drive from either the opponent's hips, side or armpit. 
If hips aren't engaged then they should be used for repeated hits, alternatively if hips are engaged but shoulder are not then they should be used.

\section*{Rolling In}
\label{drill:active_blocking:rolling_in}

The jammer rolls in with 10 feet of room. 
Before going they should attempt the following: 
\begin{itemize}
\item Juke to the line and toe stop run.
\item Juke back to the centre, engage the wall and begin jamming 
\item If all else fails, start driving to lock downt he wall, then start jamming
\end{itemize}

\end{document}
